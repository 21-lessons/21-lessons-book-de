\addpart{Fazit}
\pdfbookmark{Fazit}{conclusion}
\label{ch:conclusion}

\chapter*{Fazit}

\begin{chapquote}{Lewis Carroll, \textit{Alice im Wunderland}}
\enquote{Fang am Anfang an}, sagte der König, sehr ernst, \enquote{und mach
weiter, bis du zum Ende kommst: dann hör auf.}
\end{chapquote}

Wie eingangs erwähnt denke ich, dass jede Antwort auf die Frage
\textit{\enquote{Was hast du von Bitcoin gelernt?}} immer unvollständig sein
wird. Die Symbiose dessen was als mehrere lebende Systeme angesehen werden kann
--- Bitcoin, die Technosphäre und die Ökonomie --- ist zu sehr miteinander
verflochten, die Themen zu zahlreich und die Dinge bewegen sich zu schnell, um
jemals von einer einzigen Person vollständig verstanden zu werden.

Auch ohne alles vollständig zu verstehen und trotz all seiner Eigenheiten und
scheinbaren Mängel gibt es keine Zweifel, dass Bitcoin funktioniert. Das System
produziert etwa alle zehn Minuten einen Block. Je länger Bitcoin weiter
funktioniert, desto mehr Menschen werden sich dafür entscheiden es zu verwenden.

\begin{quotation}\begin{samepage}
\enquote{Es ist wahr, dass Dinge schön sind wenn sie funktionieren.
Kunst ist Funktion.}
\begin{flushright} -- Giannina Braschi\footnote{Giannina Braschi, \textit{Empire of Dreams} \cite{braschi2011empire}}
\end{flushright}\end{samepage}\end{quotation}

Bitcoin ist ein Kind des Internets. Es wächst exponentiell und
verwischt die Grenzen zwischen den Disziplinen. Es ist zum Beispiel nicht klar
wo das Reich der reinen Technologie endet und wo ein anderes Reich beginnt. Auch
wenn Bitcoin Computer benötigt um effizient zu funktionieren, reicht die
Informatik nicht aus um Bitcoin zu verstehen. Bitcoin ist nicht nur in Bezug auf
seine innere Funktionsweise grenzenlos, sondern auch in Bezug auf die
akademischen Disziplinen.

Wirtschaft, Politik, Spieltheorie, Geldgeschichte, Netzwerktheorie, Finanzen,
Kryptographie, Informationstheorie, Zensur, Recht und Regulierung, menschliche
Organisation, Psychologie --- all diese Disziplinen und Fachgebiete können
helfen zu verstehen wie Bitcoin funktioniert und was Bitcoin ist.

Es ist keine einzelne Erfindung die für den Erfolg verantwortlich ist. Es ist
die Kombination aus mehreren, vorher unabhängigen Teilen, die durch
spieltheoretische Anreize miteinander verstrickt wurden, welche die Revolution
von Bitcoin ausmacht. Die wunderbare Mischung aus vielen Disziplinen ist es, was
Satoshi zu einem Genie macht.

Wie jedes komplexe System muss Bitcoin Kompromisse in Bezug auf
Effizienz, Kosten, Sicherheit und viele andere Eigenschaften eingehen. So wie es
keine perfekte Lösung gibt ein Quadrat aus einem Kreis abzuleiten, wird auch
jede Lösung für die Probleme, die Bitcoin zu lösen versucht, immer unvollkommen
sein.

\begin{quotation}\begin{samepage}
\enquote{Ich glaube nicht, dass wir jemals wieder ein solides Geld haben werden
bevor wir nicht die Sache aus den Händen der Regierung nehmen, das heißt, wir
können es nicht gewaltsam aus den Händen der Regierung nehmen, alles was wir tun
können, ist auf eine schlaue Art und Weise etwas ringsherum einzuführen was sie
nicht stoppen können.}
\begin{flushright} -- Friedrich Hayek\footnote{Friedrich Hayek über Geld- und
Währungspolitik, den Goldstandard, Defizite, Inflation, und John Maynard Keynes
\url{https://youtu.be/EYhEDxFwFRU}}
\end{flushright}\end{samepage}\end{quotation}

Bitcoin ist diese schlaue Art und Weise welche ringsherum wieder solides Geld
einführt. Es tut dies, indem eine souveräne Person im Zentrum jeder Node steht,
so wie Da Vinci versucht hat das unlösbare Problem der Quadratur eines Kreises
zu lösen, indem er den vitruvianischen Menschen in sein Zentrum stellte. Ein
Netzwerk aus gleichwertigen Nodes entfernt effektiv jedes Konzept eines Zentrums
und schafft ein System, das erstaunlich antifragil und extrem schwer abzuschalten
ist. Bitcoin lebt, und sein Herzschlag wird wahrscheinlich alle unsere
überdauern.

Ich hoffe du hast diese 21 Lektionen genossen. Die vielleicht wichtigste Lektion
ist, dass Bitcoin ganzheitlich und aus mehreren Blickwinkeln betrachtet werden
sollte, wenn man etwas haben möchte das einem vollständigen Bild nahe kommt. So
wie das Entfernen eines Teils aus einem komplexen System das Ganze zerstört,
scheint die isolierte Betrachtung von Bitcoins Einzelteilen das Verständnis
davon zu beeinträchtigen. Wenn nur eine Person \enquote{Blockchain} aus ihrem
Wortschatz streichen wird um es durch \enquote{eine Kette von Blöcken} zu
ersetzen, werde ich als glücklicher Mensch sterben.

Wie auch immer, meine Reise geht auf jeden Fall weiter. Ich werde mich noch
tiefer in die Tiefen dieses Kaninchenbaues wagen und lade dich ein mir dabei zu
folgen.\footnote{\url{https://twitter.com/dergigi}}

% <!-- Twitter -->
% [dergigi]: https://twitter.com/dergigi
%
% <!-- Internal -->
% [sly roundabout way]: https://youtu.be/EYhEDxFwFRU?t=1124
% [Giannina Braschi]: https://en.wikipedia.org/wiki/Braschi%27s_Empire_of_Dreams
