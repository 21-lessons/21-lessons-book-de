\part{Technology}
\label{ch:technology}
\chapter*{Technology}

\begin{chapquote}{Lewis Carroll, \textit{Alice in Wonderland}}
\enquote{Now, I'll manage better this time} she said to herself, and began by taking
the little golden key, and unlocking the door that led into the garden
\end{chapquote}

\textit{Golden keys}, clocks which only work by chance, races to solve
strange riddles, and builders that don't have faces or names. What sounds like
fairy tales from Wonderland is daily business in the world of Bitcoin.

As we explored in Chapter~\ref{ch:economics}, large parts of the current financial
system are systematically broken. Like Alice, we can only hope to manage better
this time. But, thanks to a pseudonymous inventor, we have incredibly
sophisticated technology to support us this time around: Bitcoin.

Solving problems in a radically decentralized and adversarial environment
requires unique solutions. What would otherwise be trivial problems to solve
are everything but in this strange world of nodes. Bitcoin relies on strong
cryptography for most solutions, at least if looked at through the lens of
technology. Just how strong this cryptography is will be explored in one of the
following lessons.

\textit{Cryptography} is what Bitcoin uses to remove trust in authorities.
Instead of relying on centralized institutions, the system relies on the final
authority of our universe: physics. Some grains of trust still remain, however.
We will examine these grains in the second lesson of this chapter.

~

Part~\ref{ch:technology} -- Technology:

\begin{enumerate}
  \item Strength in numbers
  \item Reflections on \enquote{Don't Trust, Verify}
  \item Telling time takes work
  \item Move slowly and don't break things
  \item Privacy is not dead
  \item Cypherpunks write code
  \item Metaphors for Bitcoin's future
\end{enumerate}

The last couple of lessons explore the ethos of technological development in
Bitcoin, which is arguably as important as the technology itself. Bitcoin is not
the next shiny app on your phone. It is the foundation of a new economic
reality, which is why Bitcoin should be treated as nuclear-grade financial
software.

Where are we in this financial, societal, and technological revolution? Networks
and technologies of the past may serve as metaphors for Bitcoins future, which
are explored in the last lesson of this chapter.

Once more, strap in and enjoy the ride. Like all exponential technologies, we
are about to go parabolic.
