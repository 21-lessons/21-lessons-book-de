\part{Technologie}
\label{ch:technology}
\chapter*{Technologie}

\begin{chapquote}{Lewis Carroll, \textit{Alice im Wunderland}}
\enquote{Diesmal werde ich es besser machen}, sagte sie zu sich selbst und nahm
den kleinen goldenen Schlüssel und öffnete die Tür, die in den Garten führte.
\end{chapquote}

Goldene Schlüssel, Uhren die nur durch Zufall funktionieren, Wettrennen um das
Lösen von seltsamen Rätseln und Erfinder die weder Gesichter noch Namen haben.
Was wie ein Märchen aus dem Wunderland klingt gehört in der Welt von Bitcoin zum
Alltagsgeschäft.

In Kapitel~\ref{ch:economics}konnten wir sehen, dass große Teile des derzeitigen
Finanzsystems kaputt sind. Wie auch Alice können wir nur hoffen es diesmal
besser zu machen. Aber dank eines pseudonymen Erfinders haben wir diesmal eine
unglaublich ausgeklügelte Technologie die uns dabei unterstützt: Bitcoin.

Die Lösung von Problemen in einem radikal dezentralen und gegensätzlichem Umfeld
erfordert einzigartige Lösungen. Was eigentlich triviale Probleme wären, ist
alles andere als trivial wenn es um die Welt der Netzwerk-Knoten (“nodes”) geht.
Bitcoin setzt bei den meisten Lösungen auf starke Kryptographie – zumindest
durch die Linse der Technologie betrachtet. Wie stark diese Kryptographie ist,
wird in einer der folgenden Lektionen genauer untersucht.

Kryptographie ist die Waffe die Bitcoin einsetzt, um das Vertrauen in die
(staatlichen) Behörden zu stören. Anstatt sich auf zentralisierte Institutionen
zu verlassen, verlässt sich das System auf die endgültige Autorität unseres
Universums: Die Physik. Einige Punkte bei denen Vertrauen notwendig ist
existieren noch immer. Wir werden diese Punkte in der zweiten Lektion dieses
Kapitels untersuchen.
~

\begin{samepage}
Part~\ref{ch:technology} -- Technology:

\begin{enumerate}
  \setcounter{enumi}{14}
  \item Stärke in Zahlen
  \item Überlegungen zu \enquote{Don't Trust, Verify}
  \item Die Zeit zu bestimmen erfordert Arbeit
  \item Beweg dich langsam und mach nichts kaputt
  \item Privatsphäre ist nicht tot
  \item Cypherpunks schreiben Code
  \item Metaphern für Bitcoins Zukunft
\end{enumerate}
\end{samepage}

Die letzten paar Lektionen befassen sich mit dem Ethos der technologischen
Entwicklung von Bitcoin, der wohl genauso wichtig ist, wie die Technologie
selbst. Bitcoin ist nicht die nächste glänzende App auf Ihrem Handy. Es ist die
Grundlage für eine neue wirtschaftliche Realität, weshalb Bitcoin wie eine
atomwaffenfähige Finanzsoftware behandelt werden sollte.

Wo stehen wir in dieser finanziellen, gesellschaftlichen und technologischen
Revolution? Netzwerke und Technologien der Vergangenheit können als Metaphern
für die Zukunft von Bitcoin dienen, diese werden in der letzten Lektion dieses
Kapitels untersucht.

Schnallt euch noch einmal an und genießt die Fahrt. Wie alle exponentiellen
Technologien stehen wir kurz davor parabolisch steil zu gehen.
