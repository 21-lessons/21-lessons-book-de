\part{Ökonomie}
\label{ch:economics}
\chapter*{Ökonomie}

\begin{chapquote}{Lewis Carroll, \textit{Alice im Wunderland}}
\enquote{Beim Eingang zum Garten stand ein großer Rosenstock; er trug weiße
Rosen, aber drei Gärtner waren eifrig damit beschäftigt, sie rot anzustreichen.
Alice kam das sehr merkwürdig vor\ldots}
\end{chapquote}

Geld wächst nicht auf Bäumen. Zu glauben, dass es das tut, ist töricht und
unsere Eltern sorgen dafür, dass wir das wissen, indem sie diese Redewendung wie
ein Mantra wiederholen. Wir werden ermutigt Geld sinnvoll zu verwenden, es nicht
leichtfertig auszugeben und es in guten Zeiten zu sparen, um uns durch die
schlechten zu helfen. Schließlich wächst Geld nicht auf den Bäumen.

Bitcoin lehrte mich mehr über Geld als ich je dachte wissen zu müssen. Durch
Bitcoin war ich gezwungen mich mit der Geschichte des Geldes, dem Bankwesen,
verschiedenen Wirtschaftsschulen und vielen andereren Dingen zu beschäftigen.
Der Versuch Bitcoin zu verstehen führte mich auf eine Vielzahl von Wegen -
einige von diesen möchte ich hier umschreiben.

Der erste Teil war philosophischen Themen gewidmet. Die nächsten sieben
Lektionen werden sich mit Geld und Wirtschaft befassen.

~

\begin{samepage}
Teil~\ref{ch:economics} -- Ökonomie:

\begin{enumerate}
  \setcounter{enumi}{7}
  \item Finanzielle Unwissenheit
  \item Inflation
  \item Wert
  \item Geld
  \item Die Geschichte und der Untergang des Geldes
  \item Der Wahnsinn des Teilreserve-Systems
  \item Solides Geld
\end{enumerate}
\end{samepage}

Auch hier werde ich wieder nur an der Oberfläche kratzen können. Bitcoin ist
nicht nur anspruchsvoll, sondern auch breit gefächert und tiefgründig, was es
unmöglich macht alle relevanten Themen in einer einzigen Lektion, einem
Aufsatz, Artikel oder Buch zu behandeln. Ich bezweifle, dass dies überhaupt
möglich ist.

Bitcoin ist eine neue Form des Geldes, bei der das Erlernen von Ökonomie von
höchster Bedeutung für das Verständnis ist. Im Umgang mit der Natur des
menschlichen Handelns und den Wechselwirkungen der Wirtschaftsakteure ist die
Ökonomie wahrscheinlich eines der größten und verschwommensten Teile des
Bitcoin-Puzzles.

Auch diese Lektionen sind eine Erkundung der verschiedenen Dinge, die ich von
Bitcoin gelernt habe. Sie sind ein persönliches Spiegelbild meiner Reise durch
den Kaninchenbau. Da ich keinen ökonomischen Bildungshintergrund habe, befinde ich mich
definitiv außerhalb meiner Komfortzone und bin mir äußerst bewusst, dass jedes
Verständnis welches ich haben könnte unvollständig ist. Ich werde mein Bestes
tun, um zu skizzieren was ich gelernt habe, auch wenn ich Gefahr laufe mich
selbst zum Narren zu machen. Noch immer versuche ich die Frage zu beantworten:
\textit{\enquote{Was hast du von Bitcoin gelernt?}}

Nach sieben Lektionen, die durch die Linse der Philosophie betrachtet wurden,
lass uns nun die Linse der Ökonomie verwenden, um sieben weitere zu betrachten.
Die Economy-Klasse ist alles was ich diesmal anbieten kann. Endstation:
\textit{Solides Geld}.

% [the question]: https://twitter.com/arjunblj/status/1050073234719293440
