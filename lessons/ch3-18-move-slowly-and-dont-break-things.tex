\chapter{Gut Ding braucht Weile}
\label{les:18}

\begin{chapquote}{Lewis Carroll, \textit{Alice im Wunderland}}
So wand sich das Boot langsam entlang, unter dem hellen Sommertag, mit seiner
fröhlichen Crew und Musik aus Stimmen und Gelächter\ldots
\end{chapquote}

Es mag ein abgedroschenes Mantra sein, aber \enquote{\textit{move fast and break
things}} (\enquote{schnell bewegen und Dinge zerbrechen}) ist immer noch das, was
die Technikwelt antreibt. Die Idee, dass es keine Rolle spielt, ob man die Dinge
beim ersten Mal richtig macht oder nicht, ist ein Grundpfeiler der
\enquote{scheitere früh und oft} Mentalität. Erfolg wird an Wachstum gemessen,
so lange man wächst ist alles in Ordnung. Wenn etwas anfangs nicht funktioniert,
passe dich an und verändere etwas. Mit anderen Worten: Wirf genug Scheiße an die
Wand und sieh zu was kleben bleibt.

Bitcoin ist da ganz anders. Es ist schon vom Design her anders. Es ist aus
reiner Notwendigkeit anders. Wie Satoshi betonte, gab es bereits viele Versuche
E-Geld zu erschaffen, aber all diese Versuche sind gescheitert weil es einen
Kopf gab, der abgeschnitten werden konnte. Die Neuerung an Bitcoin ist, dass es
ein Monster ohne Kopf ist.

\begin{quotation}\begin{samepage}
\enquote{Viele Leute lehnen elektronische Währungen automatisch als
hoffnungslosen Fall ab, weil alle Unternehmen seit den 90er Jahren daran
scheiterten. Ich hoffe es ist offensichtlich, dass es nur die zentrale Steuerung
dieser Systeme war die sie zum Scheitern brachte.} \begin{flushright} -- Satoshi
Nakamoto\footnote{Satoshi Nakamoto, in  Antwort auf Sepp Hasslberger
\cite{satoshi-centralized-nature}}
\end{flushright}\end{samepage}\end{quotation}

Eine Folge dieser radikalen Dezentralisierung ist ein radikaler Widerstand gegen
Veränderungen. \enquote{Move fast and break things} funktioniert nicht und wird
vor allem auf der Grundebene (\enquote{base layer}) von Bitcoin nie
funktionieren. Auch wenn es wünschenswert wäre, wäre es nicht möglich ohne
\textit{alle} davon zu überzeugen ihr individuelles Verhalten zu ändern. Das ist
verteilter Konsensus. Das ist die Natur von Bitcoin.

\begin{quotation}\begin{samepage}
\enquote{Die Natur von Bitcoin ist so, dass nach der Veröffentlichung der
Version 0.1 das Kerndesign für den Rest seines Lebens in Stein gemeißelt wurde.}
\begin{flushright} -- Satoshi Nakamoto\footnote{Satoshi Nakamoto, in  Antwort
auf Gavin Andresen \cite{satoshi-centralized-nature}}
\end{flushright}\end{samepage}\end{quotation}

Dies ist eine der vielen paradoxen Eigenschaften von Bitcoin. Wir kamen doch
alle zu der Annahme, dass alles was Software ist leicht geändert werden kann.
Aber die Natur dieses Monsters macht es verdammt hart es zu verändern.

Wie Hasu in \enquote{Unpacking Bitcoin’s Social Contract}~\cite{social-contract}
zeigt, ist eine Änderung der Regeln von Bitcoin nur möglich, wenn eine Änderung
\textit{vorgeschlagen} wird und alle Benutzer von Bitcoin davon
\textit{überzeugt} werden diese Änderung zu übernehmen. Dies macht Bitcoin sehr
widerstandsfähig gegen Veränderungen auch wenn es sich \enquote{nur} um Software
handelt.

Diese Widerstandsfähigkeit ist eine der wichtigsten Eigenschaften von Bitcoin.
Kritische Softwaresysteme müssen antifragil sein, was das Zusammenspiel von
Bitcoins sozialer Ebene und seiner technischen Ebene garantiert. Währungssysteme
sind von Natur aus feindselig und wie wir seit Jahrtausenden wissen, sind solide
Grundlagen in einem feindseligen Umfeld unerlässlich.

\begin{quotation}\begin{samepage}
\enquote{Als nun ein Platzregen fiel und die Wasser kamen und die Winde wehten
und stießen an das Haus, fiel es doch nicht ein; denn es war auf Fels
gegründet.}
\begin{flushright} -- Matthäus 7:24--27
\end{flushright}\end{samepage}\end{quotation}

In diesem Gleichnis von den Weisen und den törichten Bauherren ist Bitcoin nicht
das Haus -- es ist der Fels. Unveränderlich und unbeweglich als Grundlage für
ein neues Finanzsystem.

Genau wie Geologen wissen, dass sich Gesteinsformationen ständig bewegen und
entwickeln, kann man auch sehen, dass Bitcoin sich bewegt und entwickelt.
Man muss nur wissen wo man hinschauen muss und wie man es erkennt.

Die Einführung von Pay-to-Script-Hash\footnote{Pay-to-Script-Hash (P2SH)
Transaktionen wurden in BIP 16 standardisiert. Sie erlauben es Transaktionen an
Script-Hashes (Adressen welche mit einer 3 beginnen) zu senden, anstatt an
Public-Key-Hashes (Adressen welche mit einer 1 beginnen).~\cite{btcwiki:p2sh}}
und SegWit\footnote{Segregated Witness ist ein
implementiertes Protokoll-Upgrade welches vor Transaktions-Formbarkeit
(\textit{transaction malleability}) schützt und die effektive Blockgröße
erweitert.~\cite{btcwiki:segwit}} ist ein Beweis dafür, dass die Regeln von
Bitcoin geändert werden können, wenn genügend Benutzer davon überzeugt sind,
dass die Annahme dieser Änderung dem Netzwerk zugute kommt. Letzteres
ermöglichte die Entwicklung des
Lightning-Netzwerkes\footnote{\url{https://lightning.network/}}, das eines der
Häuser ist, das auf dem soliden Fundament von Bitcoin gebaut wird. Zukünftige
Upgrades wie Schnorr-Signaturen~\cite{bip:schnorr} werden die Effizienz und den
Datenschutz verbessern, ebenso wie Skripte (\textit{smart contracts}), die sich
dank Taproot~\cite{taproot} nicht von regulären Transaktionen unterscheiden
werden. Weise Bauherren bauen in der Tat auf einem soliden Fundament auf.

Satoshi war nicht nur technisch ein weiser Baumeister. Er verstand auch, dass es
notwendig sein würde ideologisch sinnvolle Entscheidungen zu treffen.

\begin{quotation}\begin{samepage}
\enquote{Open Source bedeutet, dass jeder den Code selbstständig überprüfen
kann. Wenn es sich um eine geschlossene Software handeln würde, könnte niemand
die Sicherheit überprüfen. Ich denke es ist wichtig, dass ein solches Programm
Open Source ist.}
\begin{flushright} -- Satoshi Nakamoto\footnote{Satoshi Nakamoto, in einer
Antwort auf eine Frage von SmokeTooMuch \cite{satoshi-open-source}}
\end{flushright}\end{samepage}\end{quotation}

Offenheit ist für Sicherheit von größter Bedeutung und liegt im Wesen von Open
Source und der Bewegung für freie Software. Wie Satoshi betonte, müssen sichere
Protokolle und der Code der diese Protokolle implementiert, offen sein --- es
gibt keine Sicherheit durch Obskurität. Ein weiterer Vorteil ist wiederum die
Dezentralisierung: Code der frei ausgeführt, studiert, modifiziert, kopiert und
verteilt werden kann, sorgt für eine weite Verbreitung.

Die radikal dezentrale Natur von Bitcoin ist es, die es langsam und bewusst
bewegen lässt. Ein Netzwerk von Knoten, die jeweils von einer souveränen, also
eigenständigen Person betrieben werden, ist von Natur aus resistent gegen
Veränderungen — ob bösartig oder nicht. Da es keine Möglichkeit gibt den
Benutzern Updates aufzuzwingen, besteht der einzige Weg Änderungen einzuführen
darin, jede einzelne dieser Personen langsam davon zu überzeugen die Änderung
vorzunehmen. Dieser dezentrale Prozess der Einführung und Bereitstellung von
Änderungen macht das Netzwerk unglaublich widerstandsfähig gegen bösartige
Änderungen. Es ist auch das, was die Behebung von Bugs und sonstigen Problemen
schwieriger macht als in einer zentralisierten Umgebung, weshalb alle
Beteiligten alles daran setzen nichts zu zerstören.

\paragraph{Bitcoin lehrte mich, dass langsamer Fortschritt eines seiner Features
und kein Bug ist.}

% ---
%
% #### Through the Looking-Glass
%
% - [Lesson 1: Immutability and Change][lesson1]
%
% #### Down the Rabbit Hole
%
% - [Unpacking Bitcoin's Social Contract] by Hasu
% - [Schnorr signatures BIP][Schnorr signatures] by Pieter Wuille
% - [Taproot proposal][Taproot] by Gregory Maxwell
% - [P2SH][pay to script hash], [SegWit][segregated witness] on the Bitcoin Wiki
% - [Parable of the Wise and the Foolish Builders][Matthew 7:24--27] on Wikipedia
%
% <!-- Down the Rabbit Hole -->
% [lesson1]: {{ '/bitcoin/lessons/ch1-01-immutability-and-change' | absolute_url }}
%
% [Unpacking Bitcoin's Social Contract]: https://uncommoncore.co/unpacking-bitcoins-social-contract/
% [Matthew 7:24--27]: https://en.wikipedia.org/wiki/Parable_of_the_Wise_and_the_Foolish_Builders
% [pay to script hash]: https://en.bitcoin.it/wiki/Pay_to_script_hash
% [segregated witness]: https://en.bitcoin.it/wiki/Segregated_Witness
% [lightning network]: https://lightning.network/
% [Schnorr signatures]: https://github.com/sipa/bips/blob/bip-schnorr/bip-schnorr.mediawiki#cite_ref-6-0
% [Taproot]: https://lists.linuxfoundation.org/pipermail/bitcoin-dev/2018-January/015614.html
%
% <!-- Wikipedia -->
% [alice]: https://en.wikipedia.org/wiki/Alice%27s_Adventures_in_Wonderland
% [carroll]: https://en.wikipedia.org/wiki/Lewis_Carroll
