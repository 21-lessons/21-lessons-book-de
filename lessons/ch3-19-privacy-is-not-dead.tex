\addchap{Lesson 19: Privacy is not Dead}
\label{les:19}

\begin{chapquote}{Lewis Carroll, \textit{Alice in Wonderland}}
The players all played at once without waiting for turns, and quarrelled all
the while at the tops of their voices, and in a very few minutes the Queen was
in a furious passion, and went stamping about and shouting ``off with his
head!'' of ``off with her head!'' about once in a minute.
\end{chapquote}

If pundits are to believed, privacy has been dead [since the 80ies]. The
pseudonymous invention of Bitcoin and other events in recent history
show that this is not the case. Privacy is alive, even though it is by
no means easy to escape the surveillance state.

Satoshi went through great lengths to cover up his tracks and conceal
his identity. Ten years later, it is still unknown if Satoshi Nakamoto
was a single person, a group of people, male, female, or a
[time-traveling AI] which bootstrapped itself to take over the world.
Conspiracy theories aside, Satoshi chose to identify himself to be a
Japanese male, which is why I don't assume but respect his chosen gender
and refer to him as *he*.

% 

Whatever his real identity might be, Satoshi was successful in hiding
it. He set an encouraging example for everyone who wishes to remain
anonymous: it is possible to have privacy online.

\begin{quotation}
``Encryption works. Properly implemented strong crypto systems are one
of the few things that you can rely on.''
\end{quotation}
% > <cite>[Edward Snowden]</cite>

Satoshi wasn't the first pseudonymous or anonymous inventor, and he
won't be the last. Some have directly imitated this pseudonymous
publication style, like Tom Elvis Yedusor of [MimbleWimble] fame, while
others have published advanced mathematical proofs while remaining
completely [anonymous].

It is a strange new world we are living in. A world where identity is
optional, contributions are accepted based on merit, and people can
collaborate and transact freely. It will take some adjustment to get
comfortable with these new paradigms, but I strongly believe that all of
this has the potential to change the world for the better.

We should all remember that privacy is a [fundamental human right]. And
as long as people exercise and defend these rights the battle for
privacy is far from over. Bitcoin taught me that privacy is not dead.

% ---
%
% #### Down the Rabbit Hole
%
% - [Universal Declaration of Human Rights][fundamental human right] by the United Nations
% - [A lower bound on the length of the shortest superpattern][anonymous] by Anonymous 4chan Poster, Robin Houston, Jay Pantone, and Vince Vatter
%
% [since the 80ies]: https://books.google.com/ngrams/graph?content=privacy+is+dead&year_start=1970&year_end=2019&corpus=15&smoothing=3&share=&direct_url=t1%3B%2Cprivacy%20is%20dead%3B%2Cc0
% [time-traveling AI]: https://blockchain24-7.com/is-crypto-creator-a-time-travelling-ai/
% ["I am not Dorian Nakamoto."]: http://p2pfoundation.ning.com/forum/topics/bitcoin-open-source?commentId=2003008%3AComment%3A52186
% [Edward Snowden]: https://www.theguardian.com/world/2013/jun/17/edward-snowden-nsa-files-whistleblower
% [MimbleWimble]: https://github.com/mimblewimble/docs/wiki/MimbleWimble-Origin
% [anonymous]: https://oeis.org/A180632/a180632.pdf
% [fundamental human right]: http://www.un.org/en/universal-declaration-human-rights/
%
% <!-- Wikipedia -->
% [alice]: https://en.wikipedia.org/wiki/Alice%27s_Adventures_in_Wonderland
% [carroll]: https://en.wikipedia.org/wiki/Lewis_Carroll
