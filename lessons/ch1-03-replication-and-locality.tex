\chapter{Replikation und Lokalität}
\label{les:3}

\begin{chapquote}{Lewis Carroll, \textit{Alice im Wunderland}}
Demnächst kam eine ärgerliche Stimme -- die des Kaninchens -- \enquote{Pat! Pat! wo bist du?}
\end{chapquote}

Die Quantenmechanik außen vor gelassen, ist die Lokalisierung kein echtes
Problem in unserer physischen Welt. Die Frage \textit{\enquote{Wo ist X?}} kann
sinnvoll beantwortet werden, egal ob X eine Person oder ein Objekt ist. In der
digitalen Welt ist die Frage, \textit{wo} man sich befindet bereits schwierig,
aber nicht unmöglich zu beantworten. Wo sind deine E-Mails wirklich? Eine
schlechte Antwort wäre \enquote{in der Cloud}, da diese nur der Computer eines
anderen ist. Dennoch, wenn Sie jedes Speichermedium das Ihre Mails aufzeichnet
finden wollten, könnten Sie diese theoretisch finden.

Bei Bitcoin ist die Frage nach dem \enquote{Wo} wirklich knifflig. Wo genau sind
deine Bitcoins?

\begin{quotation}\begin{samepage}
\enquote{
Ich öffnete die Augen sah mich um und stellte das Unvermeidliche, das
Traditionelle, die bedauerlicherweise abgedroschene postoperative Frage:
\enquote{Wo bin ich?}
}
\begin{flushright} -- Daniel Dennett\footnote{Daniel Dennett, \textit{Where Am I?}~\cite{where-am-i}}
\end{flushright}\end{samepage}\end{quotation}

Das Problem besteht in zweifacher Hinsicht: Erstens wird der
\enquote{distributed Ledger} (die \enquote{verteilte Kontenübersicht}) durch
vollständige Replikation verteilt, d.h. diese Kontenübersicht ist überall.
Zweitens gibt es keine Bitcoins. Nicht nur physisch, sondern auch technisch
gesehen nicht.

Bitcoin behält den Überblick über einen Satz von \enquote{unspent transaction
outputs} ohne sich jemals auf eine Entität beziehen zu müssen, die einen Bitcoin
repräsentiert. Die Existenz eines Bitcoin wird abgeleitet aus der Menge der
\enquote{unspent transaction outputs}. Jeder Eintrag mit 100 Millionen
Basiseinheiten wird als Bitcoin bezeichnet.

\begin{quotation}\begin{samepage}
\enquote{Wo ist es in diesem Moment, im Transit? [\ldots] Erstens, es gibt keine
Bitcoins. Es gibt sie einfach nicht. Sie existieren nicht. Es gibt
Ledger-Einträge in einem Ledger, der gemeinsam genutzt wird [\ldots] Sie
existieren nicht an einem physischen Ort. Der Ledger existiert im Wesentlichen
an jedem physischen Ort. Geographie macht hier keinen Sinn --- sie wird dir
nicht helfen dein Regelwerk festzulegen.}
\begin{flushright} -- Peter Van Valkenburgh\footnote{Peter Van Valkenburgh zu Gast bei dem \textit{What Bitcoin Did} Podcast, Episode 49 \cite{wbd049}}
\end{flushright}\end{samepage}\end{quotation}

Also was besitzt du eigentlich wenn du sagst: \textit{\enquote{Ich habe einen
Bitcoin}}, wenn es keine Bitcoins gibt? Nun erinnerst du dich an all diese
komischen Worte, die du bei der Einrichtung deiner Wallet aufschreiben musstest?
Wie sich herausstellt, sind dies die Zauberworte und das was du besitzt ist ein
Zauber mit dem du Einträge in das Bitcoin-Kontenbuch machen kannst — die
Schlüssel um Bitcoins zu \enquote{bewegen}. Deshalb sind deine privaten
Schlüssel im Grunde genommen deine Bitcoins. Wenn du denkst, dass ich mir das
alles ausgedacht habe, kannst du mir gerne deine privaten Schlüssel schicken.

\paragraph{Bitcoin hat mir beigebracht, dass Lokalität eine knifflige
Angelegenheit ist.}

% ---
%
% #### Through the Looking-Glass
%
% - [The Magic Dust of Cryptography: How digital information is changing our society][a magic spell]
%
% #### Down the Rabbit Hole
%
% - [Where Am I?][Daniel Dennett] by Daniel Dennett
% - 🎧 [Peter Van Valkenburg on Preserving the Freedom to Innovate with Public Blockchains][wbd049] WBD #49 hosted by Peter McCormack
%
% <!-- Through the Looking-Glass -->
% [a magic spell]: 
%
% <!-- Down the Rabbit Hole -->
% [Daniel Dennett]: https://www.lehigh.edu/~mhb0/Dennett-WhereAmI.pdf
% [1st Amendment]: https://en.wikipedia.org/wiki/First_Amendment_to_the_United_States_Constitution
% [wbd049]: https://www.whatbitcoindid.com/podcast/coin-centers-peter-van-valkenburg-on-preserving-the-freedom-to-innovate-with-public-blockchains
%
% <!-- Wikipedia -->
% [alice]: https://en.wikipedia.org/wiki/Alice%27s_Adventures_in_Wonderland
% [carroll]: https://en.wikipedia.org/wiki/Lewis_Carroll
