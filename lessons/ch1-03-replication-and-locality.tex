\chapter{Replication and Locality}
\label{les:3}

\begin{chapquote}{Lewis Carroll, \textit{Alice in Wonderland}}
Next came an angry voice -- the rabbit's -- \enquote{Pat, Pat! where are you?}
\end{chapquote}

Quantum mechanics aside, locality is a non-issue in the physical world.
The question \textit{\enquote{Where is X?}} can be answered in a meaningful way, no
matter if X is a person or an object. In the digital world, the question
of \textit{where} is already a tricky one, but not impossible to answer. Where
are your emails, really? A bad answer would be \enquote{the cloud}, which is
just someone else's computer. Still, if you wanted to track down every
storage device which has your emails on it you could, in theory, locate
them.

With bitcoin, the question of \enquote{where} is \textit{really} tricky. Where,
exactly, are your bitcoins?

\begin{quotation}\begin{samepage}
\enquote{I opened my eyes, looked around, and asked the inevitable, the
traditional, the lamentably hackneyed postoperative question: `Where
am l?'}
\begin{flushright} -- Daniel Dennett\footnote{Daniel Dennett, \textit{Where Am I?}~\cite{where-am-i}}
\end{flushright}\end{samepage}\end{quotation}

The problem is twofold: First, the distributed ledger is distributed by
full replication, meaning the ledger is everywhere. Second, there are no
bitcoins. Not only physically, but \textit{technically}.

Bitcoin keeps track of a set of unspent transaction outputs, without
ever having to refer to an entity which represents a bitcoin. The
existence of a bitcoin is inferred by looking at the set of unspent
transaction outputs and calling every entry with a 100 million base
units a bitcoin.

\begin{quotation}\begin{samepage}
\enquote{Where is it, at this moment, in transit? [...] First, there are no
bitcoins. There just aren't. They don't exist. There are ledger
entries in a ledger that's shared [...] They don't exist in any
physical location. The ledger exists in every physical location,
essentially. Geography doesn't make sense here --- it is not going to
help you figuring out your policy here.}
\begin{flushright} -- Peter Van Valkenburgh\footnote{Peter Van Valkenburgh on the \textit{What Bitcoin Did} podcast, episode 49 \cite{wbd049}}
\end{flushright}\end{samepage}\end{quotation}

So, what do you actually own when you say \textit{\enquote{I have a bitcoin}} if
there are no bitcoins? Well, remember all these strange words which you were
forced to write down by the wallet you used? Turns out these magic words are
what you own: a magic spell\footnote{The Magic Dust of Cryptography: How digital
information is changing our society \cite{gigi:magic-spell}} which can be used
to add some entries to the public ledger --- the keys to \enquote{move} some bitcoins.
This is why, for all intents and purposes, your private keys \textit{are} your
bitcoins. If you think I'm making all of this up feel free to send me your
private keys.

\paragraph{Bitcoin taught me that locality is a tricky business.}

% ---
%
% #### Through the Looking-Glass
%
% - [The Magic Dust of Cryptography: How digital information is changing our society][a magic spell]
%
% #### Down the Rabbit Hole
%
% - [Where Am I?][Daniel Dennett] by Daniel Dennett
% - 🎧 [Peter Van Valkenburg on Preserving the Freedom to Innovate with Public Blockchains][wbd049] WBD #49 hosted by Peter McCormack
%
% <!-- Through the Looking-Glass -->
% [a magic spell]: 
%
% <!-- Down the Rabbit Hole -->
% [Daniel Dennett]: https://www.lehigh.edu/~mhb0/Dennett-WhereAmI.pdf
% [1st Amendment]: https://en.wikipedia.org/wiki/First_Amendment_to_the_United_States_Constitution
% [wbd049]: https://www.whatbitcoindid.com/podcast/coin-centers-peter-van-valkenburg-on-preserving-the-freedom-to-innovate-with-public-blockchains
%
% <!-- Wikipedia -->
% [alice]: https://en.wikipedia.org/wiki/Alice%27s_Adventures_in_Wonderland
% [carroll]: https://en.wikipedia.org/wiki/Lewis_Carroll
