\chapter{Geld}
\label{les:11}

\begin{chapquote}{Der Alte}
\enquote{Als ich jung war, \ldots \\
Da rieb ich die Glieder mir ein, \\
Mit der Salbe hier, \\
die sie geschmeidig macht -- \\
Für zwei Groschen allein ist sie dein.}
\end{chapquote}

Was ist Geld? Obwohl wir es jeden Tag benutzen, ist diese Frage überraschend
schwer zu beantworten. Wir sind davon abhängig und wenn wir zu wenig davon haben
wird unser Leben sehr schwierig. Doch wir denken selten über die eine Sache
nach, welche angeblich die Welt regiert: Geld. Bitcoin zwang mich immer wieder
diese Frage zu beantworten: Was zur Hölle ist Geld?

In unserer \enquote{modernen} Welt werden die meisten Menschen wahrscheinlich an
Papierzettel denken wenn sie über Geld sprechen, obwohl der Großteil unseres
Geldes nur eine Nummer auf einem Bankkonto bzw. einem Bankcomputer ist. Wir
verwenden bereits Nullen und Einsen als unser Geld, also wie unterscheidet sich
das von Bitcoin? Bitcoin ist anders, weil es im Kern eine ganz andere
\textit{Art} von Geld ist, als das Geld das wir derzeit verwenden. Um dies zu
verstehen, müssen wir uns genauer ansehen was Geld eigentlich ist, wie es
entstanden ist und warum Gold und Silber für den größten Teil der
Handelsgeschichte verwendet wurden.

\newpage

Muscheln, Gold, Silber, Papier, Bitcoin. \textbf{Im Endeffekt ist Geld all das,
was Menschen als Geld verwenden}, unabhängig von Kontur, Form
oder dem Nichtvorhandensein dieser physikalischen Eigenschaften.

Geld, als Erfindung betrachtet, ist genial. Eine Welt ohne Geld ist wahnsinnig
kompliziert: Wie viele Fische kosten neue Schuhe? Wie viele Kühe braucht man um
ein Haus zu kaufen? Was, wenn ich jetzt nichts brauche, aber meine Äpfel
loswerden muss, die langsam vor sich hin faulen? Man braucht nicht viel Phantasie
um zu erkennen, dass eine Tauschwirtschaft wahnsinnig ineffizient ist.

Das Tolle an Geld ist, dass es gegen alles andere eingetauscht werden kann ---
was für eine geniale Erfindung! Wie Nick
Szabo\footnote{\url{http://unenumerated.blogspot.com/}} in \textit{Shelling Out:
The Origins of Money} \cite{shelling-out} brillant zusammenfasst haben wir
Menschen alle möglichen Dinge als Geld benutzt: Perlen aus seltenen Materialien
wie Elfenbein, Muscheln oder spezielle Knochen, verschiedene Arten von Schmuck
und später seltene Metalle wie Silber und Gold.

\begin{quotation}\begin{samepage}
\enquote{In diesem Sinne ist es eher mit einem Edelmetall zu vergleichen.
Anstatt das Angebot zu verändern, um den Wert gleich zu halten, ist die Menge
vorgegeben und der Wert ändert sich.}
\begin{flushright} -- Satoshi Nakamoto\footnote{Satoshi Nakamoto, Antwort auf eine Frage von Sepp
Hasslberger \cite{satoshi-precious-metal}}
\end{flushright}\end{samepage}\end{quotation}

Als die faulen Kreaturen die wir sind, denken wir nicht allzu viel über Dinge
nach die gut funktionieren. Für die meisten von uns funktioniert Geld ganz gut.
Wie bei unseren Autos oder Computern sind die meisten von uns nur dann gezwungen über
das Innenleben dieser Dinge nachzudenken, wenn sie defekt sind. Menschen die
miterlebten wie ihre Lebensersparnisse durch Hyperinflation verschwanden, kennen
den Wert von hartem Geld. Genauso wie Menschen, die ihre Freunde und Familie
durch die Gräueltaten von Nazi-Deutschland oder der UdSSR verloren haben, den
Wert von Privatsphäre kennen.

Das Besondere an Geld ist, dass es allgegenwärtig ist. Geld ist die Hälfte jeder
Transaktion, was dazu führt, dass diejenigen die für die Schaffung des Geldes
verantwortlich sind eine enorme Macht innehalten.

\begin{quotation}\begin{samepage}
\enquote{Angesichts der Tatsache, dass Geld die Hälfte jeder kommerziellen
Transaktion ausmacht und dass ganze Zivilisationen buchstäblich aufgrund der
Qualität ihres Geldes aufsteigen und abstürzen, sprechen wir von einer
gewaltigen Macht, einer Macht, die im Schutze der Nacht fliegt. Es ist die Kraft
Illusionen zu weben, die real erscheinen, solange sie andauern. Das ist der Kern
der Macht der Fed (Zentralbank).}
\begin{flushright} -- Ron Paul\footnote{Ron Paul, \textit{End the Fed} \cite{end-the-fed}}
\end{flushright}\end{samepage}\end{quotation}

Bitcoin beseitigt diese Macht auf eine friedliche Art und Weise, da die
Notwendigkeit der Geldschöpfung gewaltfrei beseitigt wird.

Geld durchlief mehrere Phasen. Die meisten Phasen waren gut. Sie haben unser
Geld auf die eine oder andere Weise verbessert. Jedoch wurde vor noch nicht
allzu langer Zeit die Funktionsweise unseres Geldes korrumpiert. Heute wird fast
unser gesamtes Geld von (Zentral)Banken \textit{aus dem Nichts}
erschaffen. Um zu verstehen wie es dazu kam, musste ich etwas über die
Geschichte und den anschließenden Untergang des Geldes erfahren.

Ob es eine Reihe von Krisen oder einfach nur umfangreiche Aufklärung und Bildung
zur Beseitigung dieser korrupten Maschinerie braucht, bleibt abzuwarten. Ich
bete zu den Göttern des gesunden Geldes, dass es letzteres sein wird.

\paragraph{Bitcoin brachte mir bei was Geld ist.}

% ---
%
% #### Down the Rabbit Hole
%
% - [End the Fed][Ron Paul] by Ron Paul
% - [Money, blockchains, and social scalability][social-scalability] by Nick Szabo
%
% [social-scalability]: https://unenumerated.blogspot.co.at/2017/02/money-blockchains-and-social-scalability.html
%
