\chapter{Money}
\label{les:11}

\begin{chapquote}{Lewis Carroll, \textit{Alice in Wonderland}}
``In my youth,\'' said the sage, as he shook his gray locks, \\
``I kept all my limbs very supple, \\
By the use of this ointment, five shillings the box -- \\
Allow me to sell you a couple.''
\end{chapquote}

What is money? We use it every day, yet this question is surprisingly
difficult to answer. We are dependent on it in ways big and small, and
if we have too little of it our lives become very difficult. Yet, we
seldom think about the thing which supposedly makes the world go round.
Bitcoin forced me to answer this question over and over again: What the
hell is money?

In our "modern" world, most people will probably think of pieces of
paper when they talk about money, even though most of our money is just
a number in a bank account. We are already using zeros and ones as our
money, so how is Bitcoin different? Bitcoin is different because at its
core it is a very different \textit{type} of money than the money we currently
use. To understand this, we will have to take a closer look at what
money is, how it came to be, and why gold and silver was used for most
of commercial history.

\begin{quotation}
``In this sense, it's more typical of a precious metal. Instead of the
supply changing to keep the value the same, the supply is
predetermined and the value changes.''
\end{quotation}
% > <cite>[Satoshi Nakamoto]</cite>

Seashells, gold, silver, paper, bitcoin. In the end, \textbf{money is whatever
people use as money}, no matter its shape and form, or lack thereof.

Money, as an invention, is ingenious. A world without money is insanely
complicated: How many fish will buy me new shoes? How many cows will buy
me a house? What if I don't need anything right now but I need to get
rid of my soon-to-be rotten apples? You don't need a lot of imagination
to realize that a barter economy is maddeningly inefficient.

The great thing about money is that it can be exchanged for *anything
else* --- that's quite the invention! As [Nick Szabo] brilliantly
summarizes in \textit{[Shelling Out: The Origins of Money]}, we humans have
used all kinds of things as money: beads made of rare materials like
ivory, shells, or special bones, various kinds of jewelry, and later on
rare metals like silver and gold.

Being the lazy creatures we are, we don't think too much about things
which just work. Money, for most of us, works just fine. Like with our
cars or our computers, most of us are only forced to think about the
inner workings of these things if they break down. People who saw their
life-savings vanish because of hyperinflation know the value of hard
money, just like people who saw their friends and family vanish because
of the atrocities of Nazi Germany or Soviet Russia know the value of
privacy.

The thing about money is that it is all-encompassing. Money is half of
every transaction, which imbues the ones who are in charge with creating
money with enormous power.

\begin{quotation}
``Given that money is one half of every commercial transaction and that
whole civilizations literally rise and fall based on the quality of
their money, we are talking about an awesome power, one that flies
under the cover of night. It is the power to weave illusions that
appear real as long as they last. That is the very core of the
Fed's power.''
\end{quotation}
% > <cite>[Ron Paul]</cite>

Bitcoin peacefully removes this power, since it does away with money
creation and it does so without the use of force.

Money went through multiple iterations. Most iterations were good. They
improved our money in one way or another. Very recently, however, the
inner workings of our money got corrupted. Today, almost all of our
money is simply created \textit{out of thin air} by the powers that be. To
understand how this came to be I had to learn about the history and
subsequent downfall of money.

If it will take a series of catastrophes or simply a monumental
educational effort to correct this corruption remains to be seen. I pray
to the gods of sound money that it will be the latter.

\paragraph{Bitcoin taught me what money is.}

% ---
%
% #### Down the Rabbit Hole
%
% - [End the Fed][Ron Paul] by Ron Paul
% - [Shelling Out: The Origins of Money] by Nick Szabo
% - [Money, blockchains, and social scalability][social-scalability] by Nick Szabo
%
% [Satoshi Nakamoto]: http://p2pfoundation.ning.com/xn/detail/2003008:Comment:9562
% [Nick Szabo]: http://unenumerated.blogspot.com/
% [Shelling Out: The Origins of Money]: https://nakamotoinstitute.org/shelling-out/
% [Ron Paul]: http://endthefed.org/books/
% [social-scalability]: https://unenumerated.blogspot.co.at/2017/02/money-blockchains-and-social-scalability.html
%
% <!-- Wikipedia -->
% [alice]: https://en.wikipedia.org/wiki/Alice%27s_Adventures_in_Wonderland
% [carroll]: https://en.wikipedia.org/wiki/Lewis_Carroll
