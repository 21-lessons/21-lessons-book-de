\chapter{Unveränderlichkeit und Veränderung}
\label{les:1}

\begin{chapquote}{Alice}
\enquote{Vielleicht bin ich in der Nacht vertauscht worden? Überleg mal: war ich
noch dieselbe, als ich heute morgen aufgestanden bin? Ich glaube fast, daß ich
mich erinnern kann, mich etwas anders gefühlt zu haben. Aber wenn ich nicht
dieselbe bin, dann stellt sich die nächste Frage: Wer in aller Welt bin ich? Ah,
das ist das große Rätsel!}
\end{chapquote}

Bitcoin ist von Natur aus schwer zu beschreiben. Es ist ein neues \textit{Ding},
und jeder Versuch, einen Vergleich mit früheren Konzepten zu ziehen – sei es
durch die Bezeichnung \textit{digitales Gold} oder das \textit{Internet des
Geldes} – wird das Ganze zwangsläufig verfehlen. Was auch immer deine
Lieblingsanalogie sein mag, zwei Aspekte von Bitcoin sind absolut essentiell:
Dezentralisierung und Unveränderlichkeit.

\paragraph{}
Eine Möglichkeit Bitcoin zu betrachten ist der Einsatz als automatisierter
Gesellschaftsvertrag\footnote{Hasu, \textit{Unpacking Bitcoin's Social
Contract}~\cite{social-contract}}. Die Software ist nur ein Teil des Puzzles,
und zu hoffen, Bitcoin durch eine Änderung der Software zu ändern ist zwecklos.
Man müsste den Rest des Netzwerks davon überzeugen die Änderungen zu übernehmen,
was mehr ein psychologischer Aufwand als ein softwaretechnisches Vorgehen ist.

\paragraph{}
Das Folgende mag zunächst absurd klingen, wie so viele andere Dinge in dieser
verrückten Bitcoin-Welt. Dennoch glaube ich, dass es absolut zutreffend ist: Du
wirst Bitcoin nicht ändern, aber Bitcoin wird dich ändern.

\begin{quotation}\begin{samepage}
\enquote{Bitcoin wird uns mehr verändern als wir Bitcoin verändern werden.}
\begin{flushright} -- Marty Bent\footnote{Tales From the Crypt~\cite{tftc21}}
\end{flushright}\end{samepage}\end{quotation}

Es hat lange gedauert, bis ich die Tiefgründigkeit dieser Aussage erkannt habe.
Da Bitcoin nur Software und alles open-source ist, kann man die Dinge einfach
nach Belieben ändern, richtig? Falsch. \textit{Sehr} falsch. Wie zu erwarten
ist, wusste der Erfinder von Bitcoin dies nur allzu gut.

\begin{quotation}\begin{samepage}
\enquote{Die Beschaffenheit von Bitcoin ist
so, dass nach der Veröffentlichung der Version 0.1 das Kerndesign für den Rest
seines Lebens in Stein gemeißelt wurde.} \begin{flushright} -- Satoshi
Nakamoto\footnote{BitcoinTalk forum post: `Re:
Transactions and Scripts\ldots'~\cite{satoshi-set-in-stone}}
\end{flushright}\end{samepage}\end{quotation}

Viele Menschen haben versucht, die Charakteristika von Bitcoin zu verändern.
Bislang sind alle von ihnen gescheitert. Während es ein endloses Meer von \enquote{\textit{Forks}}
(Abspaltungen) und Altcoins gibt, macht das Bitcoin-Netzwerk seine Sache immer
noch genau so wie damals als der erste Bitcoin-Knoten online ging. Die Altcoins
werden auf lange Sicht keine Rolle spielen. Die Abspaltungen werden irgendwann
verhungern und aussterben. Bitcoin ist was zählt. Solange sich unser
grundlegendes Verständnis von Mathematik und/oder Physik nicht ändert, wird es
dem “Bitcoin-Honigdachs” weiterhin völlig egal sein was um ihn herum geschieht.

\begin{quotation}\begin{samepage}
\enquote{Bitcoin ist das erste Exemplar einer neuen Lebensform. Sie lebt und
atmet im Internet. Sie lebt, weil sie die Menschen dafür bezahlen kann, sie am
Leben zu erhalten. [\ldots] Sie kann nicht verändert werden. Man kann nicht
gegen sie argumentieren. Sie kann nicht manipuliert werden. Sie kann nicht
kaputt gemacht werden. Sie kann nicht gestoppt werden. [\ldots] Wenn ein
Atomkrieg die Hälfte unseres Planeten zerstören würde, würde sie unversehrt
weiterleben}
\begin{flushright} -- Ralph Merkle\footnote{DAOs, Democracy and
Governance,~\cite{merkle-dao}}
\end{flushright}\end{samepage}\end{quotation}

Der Herzschlag des Bitcoin-Netzwerks wird alle unsere überdauern.

~

Diese Erkenntnis veränderte mich mehr als sich die vergangenen Blöcke der
Bitcoin Blockchain jemals verändern werden. Es änderte meine Zeitpräferenz, mein
Verständnis von Wirtschaft, meine politischen Ansichten und vieles mehr.
Verdammt, es verändert sogar die Ernährung mancher Menschen\footnote{Inside the
World of the Bitcoin Carnivores,~\cite{carnivores}}. Wenn dir das alles verrückt
vorkommt, bist du in guter Gesellschaft. All das ist verrückt, und doch passiert
es.

~

\paragraph{Bitcoin hat mir beigebracht, dass es sich nicht ändern wird. Ich werde es.}

% ---
%
% #### Through the Looking-Glass
%
% - [Bitcoin's Gravity: How idea-value feedback loops are pulling people in][gravity]
% - [Lesson 18: Move slowly and don't break things][lesson18]
%
% #### Down the Rabbit Hole
%
% - [Unpacking Bitcoin's Social Contract][automated social contract]: A framework for skeptics by Hasu
% - [DAOs, Democracy and Governance][Ralph Merkle] by Ralph C. Merkle
% - [Marty's Bent][bent]: A daily newsletter highlighting signal in Bitcoin by Marty Bent
% - [Technical Discussion on Bitcoin's Transactions and Scripts][Satoshi Nakamoto] by Satoshi Nakamoto, Gavin Andresen, and others
% - [Inside the World of the Bitcoin Carnivores][carnivores]: Why a small community of Bitcoin users is eating meat exclusively by Jordan Pearson
% - [Tales From the Crypt][tftc] hosted by Marty Bent
%
% <!-- Internal -->
% [gravity]: 
% [lesson18]: {{ 'bitcoin/lessons/ch3-18-move-slowly-and-dont-break-things' | absolute_url }}
%
% <!-- Further Reading -->
% [automated social contract]: https://medium.com/@hasufly/bitcoins-social-contract-1f8b05ee24a9
% [carnivores]: https://motherboard.vice.com/en_us/article/ne74nw/inside-the-world-of-the-bitcoin-carnivores
% [tftc]: https://tftc.io/tales-from-the-crypt/
% [bent]: https://tftc.io/martys-bent/
%
% <!-- Quotes -->
% [Ralph Merkle]: http://merkle.com/papers/DAOdemocracyDraft.pdf
% [Satoshi Nakamoto]: https://bitcointalk.org/index.php?topic=195.msg1611#msg1611
%
% <!-- Twitter People -->
% [Marty Bent]: https://twitter.com/martybent
%
% <!-- Wikipedia -->
% [alice]: https://en.wikipedia.org/wiki/Alice%27s_Adventures_in_Wonderland
% [carroll]: https://en.wikipedia.org/wiki/Lewis_Carroll
