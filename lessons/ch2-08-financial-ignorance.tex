\chapter{Finanzielle Unwissenheit}
\label{les:8}

\begin{chapquote}{Lewis Carroll, \textit{Alice im Wunderland}}
\enquote{Und sie werden mich für ein recht dummes kleines Mädchen halten, wenn ich sie frage! Nein, das geht auf keinen Fall; vielleicht steht es irgendwo angeschrieben.}
\end{chapquote}

Eines der überraschendsten Dinge für mich war die Menge an Wissen über Finanzen,
Ökonomie und Psychologie die erforderlich ist, um zu verstehen was auf den
ersten Blick ein rein \textit{technisches} System zu sein scheint --- ein
Computernetzwerk. Um einen kleinen Kerl mit haarigen Füßen zu paraphrasieren:
\enquote{Es ist ein gefährliches Geschäft in Bitcoin einzusteigen, Frodo. Du
liest das Whitepaper und wenn du deine Füße nicht still hältst, ist nicht klar
wohin es dich spülen wird.}

Um ein neues Währungssystem verstehen zu können, muss man sich erst mit dem
Alten vertraut machen. Ich begann sehr schnell zu erkennen, dass die finanzielle
Bildung die ich durch die Schule genossen habe praktisch nicht vorhanden war.

\paragraph{}
Wie ein Fünfjähriger begann ich mir viele Fragen zu stellen: Wie funktioniert
das Bankensystem? Wie funktioniert die Börse? Was ist Fiatgeld? Was ist
\textit{normales} Geld? Warum gibt es so viele
Schulden?\footnote{\url{https://www.usdebtclock.org/}} Wie viel Geld wird
tatsächlich gedruckt und wer entscheidet das?

\newpage

Nach einer leichten Panik über den schieren Umfang meiner Unwissenheit war ich
erleichtert, als ich erkannte, dass ich mich in guter Gesellschaft befand.

\begin{quotation}\begin{samepage}
\enquote{Ist es nicht ironisch, dass Bitcoin mir mehr über Geld beigebracht hat,
als all die Jahre in denen ich für Finanzinstitute gearbeitet habe? \ldots
einschließlich des Einstiegs in meine Karriere bei einer Zentralbank}
\begin{flushright} -- Aaron\footnote{Aaron (\texttt{@aarontaycc}, \texttt{@fiatminimalist}), Tweet vom 12. Dezember 2018~\cite{aarontaycc-tweet}}
\end{flushright}\end{samepage}\end{quotation}

\begin{quotation}\begin{samepage}
\enquote{Ich habe in den letzten drei Monaten der Kryptozeit mehr über Finanzen,
Wirtschaft, Technologie, Kryptographie, Humanpsychologie, Politik, Spieltheorie,
Gesetzgebung und mich selbst gelernt als in den letzten dreieinhalb Jahren des
College}
\begin{flushright} -- Dunny\footnote{Dunny (\texttt{@BitcoinDunny}), Tweet vom 28. November 2017~\cite{bitcoindunny-tweet}}
\end{flushright}\end{samepage}\end{quotation}

Dies sind nur zwei der vielen Geständnisse auf Twitter.\footnote{Siehe
\url{http://bit.ly/btc-learned}} Bitcoin, wie in Lektion~\ref{les:1}
beschrieben, ist ein Lebewesen. Mises argumentierte, dass auch die Wirtschaft
ein Lebewesen ist. Und wie wir alle aus eigener Erfahrung wissen, sind Lebewesen
von Natur aus schwer zu verstehen.

\begin{quotation}\begin{samepage}
\enquote{Ein Wissenschaftssystem ist nur eine Station in einer endlos
fortschreitenden Suche nach Wissen. Sie wird notwendigerweise durch die
Unzulänglichkeit beeinflusst, die jeder menschlichen Anstrengung innewohnt. Aber
diese Fakten anzuerkennen bedeutet nicht, dass die heutige Wirtschaft
rückständig ist. Es bedeutet nur, dass Wirtschaft ein Lebewesen ist — und zu
leben bedeutet sowohl Unvollkommenheit als auch Veränderung.}
\begin{flushright} -- Ludwig von Mises\footnote{Ludwig von Mises, \textit{Human Action}
\cite{human-action}}
\end{flushright}\end{samepage}\end{quotation}

Wir alle lesen in den Nachrichten über verschiedene Finanzkrisen, fragen uns wie
diese großen Rettungsaktionen funktionieren und wundern uns darüber, dass
diejenigen die für Schäden in Billionenhöhe verantwortlich sind kaum zur
Verantwortung gezogen werden. Ich habe immer noch keine definitiven Antworten,
aber zumindest beginne ich einen Einblick in die Geschehnisse der Finanzwelt zu
bekommen.

Einige Leute gehen sogar so weit, die allgemeine Unwissenheit über diese Themen
auf systemische und vorsätzliche Unwissenheit zurückzuführen. Während
Geschichte, Physik, Biologie, Mathematik und Sprachen alle Teil unserer
Ausbildung sind, wird die Welt des Geldes und der Finanzen überraschenderweise,
wenn überhaupt nur oberflächlich erklärt und gelehrt. Ich frage mich ob die
Menschen immer noch bereit wären so viele Schulden zu machen, wie sie es derzeit
tun, wenn jeder wüsste was Geld bzw. Schuldgeld ist und wie unser Finanzsystem
funktioniert. Und dann frage ich mich wie viele Schichten Aluminium man wohl für
einen effektiven Aluhut braucht. Wahrscheinlich drei.

\begin{quotation}\begin{samepage}
\enquote{Diese Unfälle, diese Rettungsaktionen sind keine Zufälle. Und es ist auch kein
Zufall, dass es in der Schule keine finanzielle Bildung gibt. [\ldots] Es ist
vorsätzlich. Genau wie vor dem Bürgerkrieg als es illegal war einen Sklaven
auszubilden, dürfen wir in der Schule nichts über Geld lernen.}
\begin{flushright} -- Robert Kiyosaki\footnote{Robert Kiyosaki, \textit{Why the Rich
are Getting Richer} (Warum die Reichen immer reicher werden) \cite{robert-kiyosaki}}
\end{flushright}\end{samepage}\end{quotation}

Wie in \enquote{Der Zauberer von Oz} wird uns gesagt, dass wir dem Mann hinter
dem Vorhang keine Aufmerksamkeit schenken sollen. Im Gegensatz zum Zauberer von
Oz haben wir jetzt aber echte
Zauberei\footnote{\url{http://bit.ly/btc-wizardry}}: ein zensurresistentes,
offenes, grenzenloses Netzwerk der Wertübertragung. Es gibt keinen Vorhang und
die Magie ist für jeden offen
sichtbar.\footnote{\url{https://github.com/bitcoin/bitcoin}}

\paragraph{Bitcoin lehrte mich hinter den Vorhang zu schauen und mich meiner
finanziellen Unwissenheit zu stellen.}

% ---
%
% #### Down the Rabbit Hole
%
% - [Human Action][Ludwig von Mises] by Ludwig von Mises
% - [Why the Rich are Getting Richer][Robert Kiyosaki] by Robert Kiyosaki
%
% [real wizardry]: https://external-preview.redd.it/8d03MWWOf2HIyKrT8ThBGO4WFv-u25JaYqhbEO9b1Sk.jpg?width=683&auto=webp&s=dc5922d84717c6a94527bafc0189fd4ca02a24bb
% [visible to anyone]: https://github.com/bitcoin/bitcoin
%
% <!-- Wikipedia -->
% [alice]: https://en.wikipedia.org/wiki/Alice%27s_Adventures_in_Wonderland
% [carroll]: https://en.wikipedia.org/wiki/Lewis_Carroll
