\chapter{Die Macht der freien Meinungsäußerung}
\label{les:6}

\begin{chapquote}{Lewis Carroll, \textit{Alice im Wunderland}}
\enquote{Bitte um Verzeihung!} sprach die Maus mit gerunzelter Stirne,
aber sehr höflich; \enquote{haben Sie gesprochen?}
\end{chapquote}

Bitcoin ist eine Idee. Eine Idee, die in ihrer jetzigen Form die Manifestation
einer Maschine trägt, die rein durch Text angetrieben wird. Jeder Aspekt von
Bitcoin ist Text: Das Whitepaper ist Text. Die Software die von den Knoten
ausgeführt wird ist Text. Das Kontenbuch ist Text. Transaktionen sind Text.
Öffentliche und private Schlüssel sind Text. Jeder Aspekt von Bitcoin ist Text
und damit gleichbedeutend mit Sprache.

\begin{quotation}\begin{samepage}
\enquote{Der Kongress soll kein Gesetz erlassen, das eine Einrichtung einer
Religion zum Gegenstand hat oder deren freie Ausübung beschränkt, oder eines,
das Rede- und Pressefreiheit oder das Recht des Volkes, sich friedlich zu
versammeln und an die Regierung eine Petition zur Abstellung von Missständen zu
richten, einschränkt.}
\begin{flushright} -- 1. Zusatzartikel zur Verfassung der Vereinigten Staaten
\end{flushright}\end{samepage}\end{quotation}

Obwohl die letzte Schlacht der \enquote{\textit{Crypto Wars}}\footnote{Als \textit{Crypto
Wars} bezeichnet man Bestrebungen der US-amerikanischen Regierung, die private
Verschlüsselung von Daten zu unterbinden. Für die unterschiedlichen Phasen
dieser Auseinandersetzung ist mitunter von Crypto Wars 2.0 und 3.0 die Rede. Die
Crypto Wars nehmen einen bedeutenden Anteil an der Geschichte der Kryptographie
ein.~\cite{eff-cryptowars}~\cite{wiki:cryptowars}} noch nicht geführt ist, wird
es sehr schwierig werden eine Idee zu kriminalisieren, vor allem eine Idee die
auf dem Austausch von Textnachrichten basiert. Jedes Mal wenn eine Regierung
versucht Text oder Sprache zu verbieten, gehen wir einen weiteren Schritt in die
Absurdität, welche zwangsläufig zu Abscheulichkeiten wie illegalen
Zahlen\footnote{Die Bezeichnung illegale Zahl wird für Zahlen verwendet, die
eine Information darstellen, deren Besitz und/oder deren Verbreitung
gesetzwidrig ist.\cite{wiki:illegal-number}} und illegalen
Primzahlen\footnote{Die Bezeichnung illegale Primzahl wird gelegentlich im
Zusammenhang mit bestimmten Primzahlen verwendet, die so konstruiert sind, dass
sie auf eine bekannte Weise in geschützte Daten umgewandelt werden können,
beispielsweise in den Quellcode eines Programms, das Kopierschutzmechanismen
oder Verschlüsselungen umgeht.\cite{wiki:illegal-prime}} führt.

Solange es einen Teil der Welt gibt in dem die Rede frei im Sinne von
\textit{Freiheit} ist, solange ist Bitcoin unaufhaltsam.

\begin{quotation}\begin{samepage}
\enquote{An keiner Stelle einer Transaktion hört Bitcoin auf \textit{Text} zu
sein. Es ist \textit{alles Text}, die ganze Zeit. [\ldots] Bitcoin ist
\textit{Text}. Bitcoin ist \textit{Sprache}. Es kann in einem freien Land wie
den USA mit garantierten unveräußerlichen Rechten und dem ersten Zusatzartikel
nicht reguliert werden. Der erste Zusatzartikel erlaubt es ausdrücklich Texte
ohne Regierungsaufsicht zu veröffentlichen.}

\begin{flushright} -- Beautyon\footnote{Beautyon, \textit{Why America can't regulate
Bitcoin} \cite{america-regulate-bitcoin}}
\end{flushright}\end{samepage}\end{quotation}

\paragraph{Bitcoin lehrte mich, dass in einer freien Gesellschaft freie
Meinungsäußerung und freie Software unaufhaltsam sind.}

% ---
%
% #### Through the Looking-Glass
%
% - [The Magic Dust of Cryptography: How digital information is changing our society][a magic spell]
%
% #### Down the Rabbit Hole
%
% - [Why America can't regulate Bitcoin][Beautyon] by Beautyon
% - [First Amendment to the United States Constitution][1st Amendment], [Crypto Wars], [illegal numbers], [illegal primes] on Wikipedia
%
% <!-- Through the Looking-Glass -->
% [a magic spell]: 
%
% <!-- Down the Rabbit Hole -->
% [1st Amendment]: https://en.wikipedia.org/wiki/First_Amendment_to_the_United_States_Constitution
% [Crypto Wars]: https://en.wikipedia.org/wiki/Crypto_Wars
% [illegal numbers]: https://en.wikipedia.org/wiki/Illegal_number
% [illegal primes]: https://en.wikipedia.org/wiki/Illegal_prime
% [Beautyon]: https://hackernoon.com/why-america-cant-regulate-bitcoin-8c77cee8d794
%
% <!-- Wikipedia -->
% [alice]: https://en.wikipedia.org/wiki/Alice%27s_Adventures_in_Wonderland
% [carroll]: https://en.wikipedia.org/wiki/Lewis_Carroll
