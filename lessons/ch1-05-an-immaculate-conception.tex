\chapter{Eine unbefleckte Empfängnis}
\label{les:5}

\begin{chapquote}{Lewis Carroll, \textit{Alice im Wunderland}}
\enquote{Ihre Köpfe sind fort,} schrien die Soldaten als Antwort\ldots
\end{chapquote}

Jeder liebt eine gute Entstehungsgeschichte. Die Entstehungsgeschichte von
Bitcoin ist faszinierend und die Details sind wichtiger als man zunächst denken
mag. Wer ist Satoshi Nakamoto? War er eine Person oder eine Gruppe von Menschen?
War er eine sie? Zeitreisende Außerirdische oder fortschrittliche KI? Seltsame
Verschwörungstheorien mal außer acht gelassen, wir werden es wahrscheinlich nie
erfahren. Und das ist wichtig.

Satoshi entschied sich für die Anonymität. Er pflanzte den Samen von Bitcoin. Er
blieb lange genug in der Nähe um sicherzustellen, dass das Netzwerk nicht schon
in den Kinderschuhen stirbt. Dann verschwand er.

Was wie ein seltsamer Anonymitätsstunt aussehen mag ist entscheidend für ein
wirklich dezentrales System. Keine zentrale Steuerung. Keine zentrale Behörde.
Kein Erfinder. Niemand der verfolgt, gefoltert oder erpresst werden kann. Eine
unbefleckte Empfängnis von Technologie.

\begin{quotation}\begin{samepage}
\enquote{Eines der besten Dinge, die Satoshi tat, war zu verschwinden.}
\begin{flushright} -- Jimmy Song\footnote{Jimmy Song, \textit{Why Bitcoin is Different} (Warum Bitcoin anders ist) \cite{bitcoin-different}}
\end{flushright}\end{samepage}\end{quotation}

\newpage

Seit der Erfindung von Bitcoin wurden Tausende andere Kryptowährungen
geschaffen. Keiner dieser Klone teilt seine Entstehungsgeschichte. Wenn man
Bitcoin ersetzen wollte, müsste man dessen Entstehungsgeschichte übertreffen. In
einem Krieg der Ideen, bestimmen Narrative das Überleben.

\begin{quotation}\begin{samepage}
\enquote{Gold wurde zuerst zu Schmuck verarbeitet und vor über 7000 Jahren für
den Tausch verwendet. Der fesselnde Glanz von Gold führte dazu, dass es als
Geschenk der Götter angesehen wurde.}
\begin{flushright} Austrian Mint\footnote{The Austrian Mint, \textit{Gold: The Extraordinary Metal} \cite{gold-gift-gods}}
\end{flushright}\end{samepage}\end{quotation}

Wie auch Gold in der Antike könnte Bitcoin als Geschenk der Götter angesehen
werden. Im Gegensatz zu Gold sind die Ursprünge Bitcoins nur allzu menschlich.
Und diesmal wissen wir, wer die Götter der Entwicklung und Instandhaltung sind:
Menschen auf der ganzen Welt, anonym oder nicht.

\paragraph{Bitcoin lehrte mich, dass Entstehungsgeschichten wichtig sind.}

% ---
%
% #### Down the Rabbit Hole
%
% - [Why Bitcoin is different][Jimmy Song] by Jimmy Song
% - [Gold: The Extraordinary Metal] by the Austrian Mint
%
% <!-- Down the Rabbit Hole -->
% [Jimmy Song]: https://medium.com/@jimmysong/why-bitcoin-is-different-e17b813fd947
% [Gold: The Extraordinary Metal]: https://www.muenzeoesterreich.at/eng/discover/for-investors/gold-the-extraordinary-metal
%
% <!-- Wikipedia -->
% [alice]: https://en.wikipedia.org/wiki/Alice%27s_Adventures_in_Wonderland
% [carroll]: https://en.wikipedia.org/wiki/Lewis_Carroll
