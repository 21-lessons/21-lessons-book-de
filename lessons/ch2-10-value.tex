\chapter{Wert}
\label{les:10}

\begin{chapquote}{Lewis Carroll, \textit{Alice im Wunderland}}
\enquote{Es war das Weiße Kaninchen, das langsam zurücktrottete und dabei
ängstlich umherschaute, als habe es etwas verloren\ldots}
\end{chapquote}

Wert ist etwas Paradoxes und es gibt mehrere Theorien\footnote{Siehe
\textit{Werttheorie} auf Wikipedia \cite{wiki:theory-of-value}} die versuchen zu
erklären warum wir gewisse Dinge höher als andere bewerten. Die Menschen sind
sich dieses Paradoxons seit Jahrtausenden bewusst. Bereits Platon schrieb in
seinem Dialog mit Euthydemus, dass wir einige Dinge wertschätzen weil sie selten
sind und nicht ausschließlich weil sie notwendig für unser Überleben sind.

\begin{quotation}\begin{samepage}
\enquote{Und wenn du besonnen bist wirst du diesen Rat auch deinen Schülern
geben --- den sie nie mit jemandem außer dir und untereinander teilen sollen.
Denn es ist das Seltene was kostbar ist, während Wasser am billigsten, aber am
besten ist, wie Pindar sagte.}
\begin{flushright} -- Platon\footnote{Platon, \textit{Euthydemus} \cite{euthydemus}}
\end{flushright}\end{samepage}\end{quotation}

Dieses klassiche Wertparadoxon\footnote{Siehe \textit{Klassisches Wertparadoxon}
auf Wikipedia \cite{wiki:paradox-of-value}} zeigt etwas Interessantes an uns
Menschen: Wir scheinen die Dinge auf einer subjektiven\footnote{Siehe
\textit{Grenznutzenschule} auf Wikipedia \cite{wiki:subjective-theory-of-value}}
Grundlage zu bewerten, tun dies aber mit bestimmten, nicht willkürlichen
Kriterien. Etwas mag für uns aus verschiedenen Gründen \textit{wertvoll} sein,
aber Dinge die wir schätzen, haben bestimmte Eigenschaften. Wenn wir etwas sehr
leicht kopieren können oder wenn es von Natur aus reichlich vorhanden ist,
schätzen wir es nicht.

Es hat den Anschein, dass wir etwas wertschätzen, weil es knapp ist (Gold,
Diamanten, Zeit), schwierig oder arbeitsintensiv zu produzieren ist, nicht
ersetzt werden kann (ein altes Foto eines geliebten Menschen), nützlich ist in
einer Weise die es uns ermöglicht Dinge zu tun, die wir sonst nicht tun könnten
oder eine Kombination mehrere Aspekte, wie bei großartigen Kunstwerken.

Bitcoin ist all das oben genannte: Es ist extrem selten (21 Millionen), immer
schwieriger zu produzieren (Halbierung der Belohnung), kann nicht ersetzt werden
(ein verlorener privater Schlüssel geht für immer verloren) und es ermöglicht
uns einige sehr nützliche Dinge zu tun. Es ist wohl das beste Instrument für den
grenzüberschreitenden Wertetransfer und praktisch resistent gegen Zensur und
Beschlagnahmung. Es ist ein Wertspeicher, der es Einzelpersonen ermöglicht ihr
Vermögen unabhängig von Banken, Regierungen oder anderen Dritten aufzubewahren.
Und es ist noch so viel mehr.

\paragraph{Bitcoin lehrte mich, dass der Wert subjektiv aber nicht willkürlich ist.}

% ---
%
% #### Down the Rabbit Hole
%
% - [Euthydemus] by Plato
% - [Theory of Value][multiple theories], [Paradox of Value][paradox of value], [Subjective Theory of Value][subjective] on Wikipedia
%
% [Euthydemus]: http://www.perseus.tufts.edu/hopper/text?doc=Perseus:text:1999.01.0178:text=Euthyd.
% [Plato]: http://www.perseus.tufts.edu/hopper/text?doc=plat.+euthyd.+304b
%
% <!-- Wikipedia -->
% [multiple theories]: https://en.wikipedia.org/wiki/Theory_of_value_%28economics%29
% [paradox of value]: https://en.wikipedia.org/wiki/Paradox_of_value
% [subjective]: https://en.wikipedia.org/wiki/Subjective_theory_of_value
% [alice]: https://en.wikipedia.org/wiki/Alice%27s_Adventures_in_Wonderland
% [carroll]: https://en.wikipedia.org/wiki/Lewis_Carroll
