\chapter{The problem of identity}
\label{les:4}

\begin{chapquote}{Lewis Carroll, \textit{Alice in Wonderland}}
  ``Who are you?'' said the caterpillar.
\end{chapquote}

Nic Carter, in an homage to Thomas Nagel's treatment of the same
question in regards to a bat, wrote an excellent piece which discusses
the following question: What is it like to be a bitcoin? He
brilliantly shows that open, public blockchains in general, and Bitcoin
in particular, suffer from the same conundrum as the Ship of
Theseus: which Bitcoin is the real Bitcoin?

\begin{quotation}
``Consider just how little persistence Bitcoin's components have. The
entire codebase has been reworked, altered, and expanded such that it
barely resembles its original version. [...] The registry of who
owns what, the ledger itself, is virtually the only persistent trait
of the network [...]
To be considered truly leaderless, you must surrender the easy
solution of having an entity that can designate one chain as the
legitimate one.''
\flushright -- Nic Carter\footnote{Nic Carter, What is it like to be a bitcoin?} % TODO cite
\end{quotation}
% > <cite>[Nic Carter][What is it like to be a bitcoin?]</cite>

It seems like the advancement of technology keeps forcing us to take
these philosophical questions seriously. Sooner or later, self-driving
cars will be faced with real-world versions of the trolley problem,
forcing them to make ethical decisions about whose lives do matter and
whose do not.

Cryptocurrencies, especially since the first contentious hard-fork,
force us to think about and agree upon the metaphysics of identity.
Interestingly, the two biggest examples we have so far have lead to two
different answers. On August 1, 2017, Bitcoin split into two camps. The
market decided that the unaltered chain is the original Bitcoin. One
year earlier, on October 25, 2016, Ethereum split into two camps. The
market decided that the \textit{altered} chain is the original Ethereum.

If properly decentralized, the questions posed by the \textit{Ship of Theseus}
will have to be answered in perpetuity for as long as these networks of
value-transfer exist.

\paragraph{Bitcoin taught me that decentralization contradicts identity.}

% ---
%
% #### Down the Rabbit Hole
%
% - [What Is It Like to be a Bat?][in regards to a bat] by Thomas Nagel
% - [What is it like to be a bitcoin?] by Nic Carter
% - [Ship of Theseus], [trolley problem] on Wikipedia
%
% [in regards to a bat]: https://en.wikipedia.org/wiki/What_Is_it_Like_to_Be_a_Bat%3F
% [What is it like to be a bitcoin?]: https://medium.com/s/story/what-is-it-like-to-be-a-bitcoin-56109f3e6753
% [Ship of Theseus]: https://en.wikipedia.org/wiki/Ship_of_Theseus
% [trolley problem]: https://en.wikipedia.org/wiki/Trolley_problem
%
% <!-- Wikipedia -->
% [alice]: https://en.wikipedia.org/wiki/Alice%27s_Adventures_in_Wonderland
% [carroll]: https://en.wikipedia.org/wiki/Lewis_Carroll
