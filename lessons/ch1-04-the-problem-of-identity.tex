\chapter{Das Problem der Identität}
\label{les:4}

\begin{chapquote}{Lewis Carroll, \textit{Alice im Wunderland}}
  \enquote{Wer bist du?} sagte die Raupe.
\end{chapquote}

Nic Carter schrieb einen hervorragenden Artikel, der als eine Hommage an Thomas
Nagels Werk \enquote{\textit{What is it like to be a bat}}, die folgende Frage behandelt:
\enquote{Wie ist es ein Bitcoin zu sein?} Er zeigt auf brillante Weise, dass
offene und öffentliche Blockchains im Allgemeinen und Bitcoin im Besonderen
unter dem gleichen Problem leiden wie das Schiff des Theseus\footnote{ Das
Schiff des Theseus (auch Theseus-Paradoxon) ist ein philosophisches Paradoxon,
das bereits in der Antike aufgezeigt wurde. Es berührt die Frage, ob ein
Gegenstand seine Identität verliert, wenn viele oder gar alle seiner Einzelteile
nacheinander ausgetauscht werden.~\cite{wiki:theseus}}: Welcher Bitcoin ist der
echte Bitcoin?

\begin{quotation}\begin{samepage}
\enquote{Man beachte, wie wenig beständig die Bestandteile von Bitcoin sind. Der
gesamte Quellcode wurde überarbeitet, geändert und erweitert, so dass er
mittlerweile kaum der ursprünglichen Version ähnelt. [\ldots] Der Ledger, in
dem steht, wer was besitzt, das Kontenbuch selbst, ist praktisch die einzige
dauerhafte Eigenschaft des Netzwerks.[\ldots]

Um als wirklich führerlos angesehen zu werden, muss man sich von der einfachen
Lösung, eine Einheit zu haben, die eine einzige Kette als die legitime bestimmen
kann, lossagen.}
\begin{flushright} -- Nic Carter\footnote{Nic Carter, \textit{What is it like to be a bitcoin?} \cite{bitcoin-identity}}
\end{flushright}\end{samepage}\end{quotation}

Es scheint als würde uns der technologische Fortschritt immer wieder dazu
zwingen, diese philosophischen Fragen ernst zu nehmen. Früher oder später werden
selbstfahrende Autos mit realen Versionen des Trolley-Problems konfrontiert
sein, die sie zwingen, ethische Entscheidungen darüber zu treffen, wessen Leben
wichtig ist und wessen nicht.

Kryptowährungen, besonders seit dem ersten umstrittenen Hardfork\footnote{nicht rückwärtskompatible Abspaltung}, zwingen uns
über die Metaphysik der Identität nachzudenken. Wir müssen uns ständig darauf
einigen welche Identität die richtige ist. Interessanterweise haben die beiden
prominentesten Beispiele zu zwei  unterschiedlichen Antworten geführt. Am 1.
August 2017 teilte sich Bitcoin in zwei Lager auf. Der Markt entschied, dass die
unveränderte Kette die richtige Bitcoin Kette ist. Ein Jahr zuvor, am 25. Oktober
2016, teilte sich Ethereum in zwei Lager auf. Der Markt entschied, dass die
\textit{veränderte} Kette das richtige Ethereum ist.

Für wirkliche Dezentralisierung müssen die Fragen des Schiffes von Theseus
solange beantwortet werden, wie diese Netzwerke des Wert-Transfers bestehen.

\paragraph{Bitcoin lehrte mich, dass sich Dezentralisierung und Identität widersprechen.}

% ---
%
% #### Down the Rabbit Hole
%
% - [What Is It Like to be a Bat?][in regards to a bat] by Thomas Nagel
% - [What is it like to be a bitcoin?] by Nic Carter
% - [Ship of Theseus], [trolley problem] on Wikipedia
%
% [in regards to a bat]: https://en.wikipedia.org/wiki/What_Is_it_Like_to_Be_a_Bat%3F
% [What is it like to be a bitcoin?]: https://medium.com/s/story/what-is-it-like-to-be-a-bitcoin-56109f3e6753
% [Ship of Theseus]: https://en.wikipedia.org/wiki/Ship_of_Theseus
% [trolley problem]: https://en.wikipedia.org/wiki/Trolley_problem
%
% <!-- Wikipedia -->
% [alice]: https://en.wikipedia.org/wiki/Alice%27s_Adventures_in_Wonderland
% [carroll]: https://en.wikipedia.org/wiki/Lewis_Carroll
