\chapter{Die Knappheit der Knappheit}
\label{les:2}

\begin{chapquote}{Alice}
\enquote{Das ist genug – ich hoffe, ich werde nicht mehr wachsen\ldots}
\end{chapquote}

Im Allgemeinen scheint der Fortschritt der Technologie Güter erschwinglicher zu
machen, so dass diese immer reichlicher vorhanden sind. Immer mehr Menschen
können jene Güter genießen, die bis dato als Luxusgüter galten. Bald werden wir
alle wie Könige leben -- die meisten von uns tun es bereits. Wie Peter Diamandis
in \textit{Abundance}~\cite{abundance} schrieb: \enquote{Technologie ist ein
Ressourcen-freisetzender Mechanismus. Was einst knapp war, ist nun reichlich
vorhanden.}

Bitcoin, an sich eine fortschrittliche Technologie, durchbricht diesen Trend und
schafft eine neue Ware, die wirklich knapp ist. Einige argumentieren sogar, dass
es eines der seltensten Dinge im Universum ist. Das Angebot kann nicht
vergrößert werden, egal wie viel Aufwand jemand betreibt, um mehr davon zu
erschaffen.

\begin{quotation}\begin{samepage}
\enquote{Nur zwei Dinge sind wirklich knapp: Zeit und Bitcoin.}
\begin{flushright} -- Saifedean Ammous\footnote{Presentation über den Bitcoin
Standard~\cite{bitcoinstandard-pres}}
\end{flushright}\end{samepage}\end{quotation}

Paradoxerweise geschieht dies durch einen Kopiervorgang. Transaktionen werden
verteilt, Blöcke werden verbreitet, das verteilte Kontenbuch
(\textit{distributed Ledger}) ist, wie der Name schon sagt, verteilt. All diese
Worte sind nur ausgefallene Formulierungen für das Kopieren. Sogar Bitcoin
selbst kopiert sich auf so viele Computer wie nur möglich, indem es einzelne
Leute dazu anregt, Fullnodes zu betreiben und neue Blöcke zu minen.

All diese Duplikation arbeitet auf wunderbare Weise zusammen, um gemeinsam
Knappheit zu erzeugen.

\paragraph{In einer Zeit des Überflusses lehrte mich Bitcoin, was echte
Knappheit ist.}

% ---
%
% #### Through the Looking-Glass
%
% - [Lesson 14: Sound money][lesson14]
%
% #### Down the Rabbit Hole
%
% - [The Bitcoin Standard: The Decentralized Alternative to Central Banking][bitcoin-standard]
% - [Abundance: The Future Is Better Than You Think][Abundance] by Peter Diamandis
% - [Presentation on The Bitcoin Standard][bitcoin-standard-presentation] by Saifedean Ammous
% - [Modeling Bitcoin's Value with Scarcity][planb-scarcity] by PlanB
% - 🎧 [Misir Mahmudov on the Scarcity of Time & Bitcoin][tftc60] TFTC #60 hosted by Marty Bent
% - 🎧 [PlanB – Modelling Bitcoin's digital scarcity through stock-to-flow techniques][slp67] SLP #67 hosted by Stephan Livera
%
% <!-- Through the Looking-Glass -->
% [lesson14]: {{ 'bitcoin/lessons/ch2-14-sound-money' | absolute_url }}
%
% <!-- Down the Rabbit Hole -->
% [Abundance]: https://www.diamandis.com/abundance
% [bitcoin-standard]: http://amzn.to/2L95bJW
% [bitcoin-standard-presentation]: https://www.bayernlb.de/internet/media/de/ir/downloads_1/bayernlb_research/sonderpublikationen_1/bitcoin_munich_may_28.pdf
% [planb-scarcity]: https://medium.com/@100trillionUSD/modeling-bitcoins-value-with-scarcity-91fa0fc03e25
% [tftc60]: https://anchor.fm/tales-from-the-crypt/episodes/Tales-from-the-Crypt-60-Misir-Mahmudov-e3aibh
% [slp67]: https://stephanlivera.com/episode/67
%
% <!-- Wikipedia -->
% [alice]: https://en.wikipedia.org/wiki/Alice%27s_Adventures_in_Wonderland
% [carroll]: https://en.wikipedia.org/wiki/Lewis_Carroll
