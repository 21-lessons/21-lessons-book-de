\chapter*{Vorwort}
\pdfbookmark{Vorwort}{foreword}

Manche nennen es eine religiöse Erfahrung. Andere nennen es Bitcoin.

Ich traf Gigi zum ersten Mal in meiner spirituellen Heimat -- Riga, Lettland --
wo die \textit{Baltic Honeybadger} Konferenz stattfand, zu der die
leidenschaftlichsten der Bitcoin-Gläubigen jedes Jahr hinpilgern. Nach einem
intensiven Gespräch zur Mittagszeit war die spirituelle Verbindung zwischen Gigi
und mir so fest in Stein gemeißelt wie eine der Bitcoin-Transaktionen, die
verarbeitet wurden, als wir uns einige Stunden zuvor das erste Mal die Hand
schüttelten.

In meinem anderen spirituellen Zuhause -- Christ Church, Oxford -- in dem ich
das Privileg hatte, meinen MBA zu machen, hatte ich meinen eigenen
\enquote{Kaninchenbau-Moment}. Wie Gigi transzendierte ich den wirtschaftlichen,
technischen und sozialen Bereich und war geistig von Bitcoin umwoben. Nachdem
ich in der November-Blase 2013 am Höhepunkt des Bullenmarktes gekauft hatte,
lernte ich viele äußerst harte Lektionen im unerbittlichen,
nicht-enden-wollenden Bärenmarkt der nächsten 3 Jahre. Diese 21 Lektionen hätten
mir in jener Zeit in der Tat sehr geholfen. Viele dieser Lektionen sind einfach
naturgegebene Wahrheiten, die für den Uneingeweihten durch einen
undurchsichtigen, fragilen Film verdeckt werden. Am Ende dieses Buches wird
diese undurchsichtige Fassade jedoch stark fragmentiert sein.

Ende August 2016, in einer kristallklaren Nacht in Oxford, nur wenige Wochen
nachdem mir das Messer durch den Hack der Bitfinex-Börse erneut ins Herz
gerammt wurde, saß ich in stiller Besinnung im großen Garten -- dem
sogenannten \textit{Master Garden} -- von Christ Church. Die Zeiten waren hart,
und dank dieser Folter, welche sich wie eine lebenslange Pein anfühlte, war ich
am Ende meiner mentalen und emotionalen Kräfte - kurz vor einem Zusammenbruch.
Nicht wegen finanzieller Verluste, sondern wegen des erdrückenden spirituellen
Verlusts fühlte ich mich in meinem Weltbild isoliert. Wenn es damals nur
Ressourcen wie diese gegeben hätte, welche mir gezeigt hätten, dass ich nicht
alleine bin. Der \textit{Master Garden} ist für mich und viele andere Menschen,
welche im Laufe der Jahrhunderte diesen Garten bestaunen durften, ein ganz
besonderer Ort. Es war dort, in diesem Garten, wo ein gewisser Charles Dodgson
(ein Mathe-Tutor am Christ Church College) eine seiner jungen Schülerinnen,
Alice Liddell (die Tochter des Dekans des Colleges), beobachtete. Dodgson,
besser bekannt unter seinem Künstlernamen \textit{Lewis Carroll}, ließ sich von
Alice und dem Garten inspirieren als er \enquote{Alice im Wunderland} schrieb.
Es war in der magischen Präsenz dieses heiligen Rasens als ich in die tiefsten
Tiefen des Bitcoin-Kaninchenbaus starrte; und Bitcoin starrte blitzend aus dem
Abgrund zurück, vernichtete meine Arroganz, und schlug mir meinen Stolz
regelrecht aus dem Gesicht. Ich kam endlich zur Ruhe, und fand meinen inneren
Frieden.

Diese 21 Lektionen entführen dich auf eine echte Bitcoin-Reise. Nicht nur eine
Reise der Philosophie, Technologie und Wirtschaft, sondern auch eine Reise der
Seele.

Wenn du tiefer in die Philosophie eintauchst, die in 7 der 21 Lektionen knapp
dargelegt ist, kannst du, mit genügend Zeit und Kontemplation, sogar den
Ursprung des Daseins verstehen. In seinen sieben Lektionen über Ökonomie zeigt
er in einfachen Worten, wie wir einer kleinen Gruppe von verrückten Hutmachern
finanziell ausgeliefert sind und wie diese es erfolgreich geschafft haben,
unsere Gedanken, Herzen und Seelen zu täuschen. Die 7 Lektionen über Technologie
zeigen die Schönheit und technologisch-darwinistische Perfektion von Bitcoin.
Als nicht-technischer Bitcoiner bieten die Lektionen einen umfassenden Überblick
über die zugrunde liegende technologische Natur von Bitcoin und in der Tat die
Natur der Technologie selbst.

Wir lieben, wir leben, wir lernen. Das ist die Essenz dieser vergänglichen
Erfahrung welche wir \enquote{das Leben} nennen. Aber was ist das Leben, außer
eine zeitgestempelte Reihenfolge von Ereignissen?

Die Eroberung des Bitcoin-Berges ist nicht einfach. Falsche Gipfel so weit das
Auge reicht, scharfe Felsen, rauhes Gebirge und überall warten Risse und Spalten
darauf dich zu verschlingen. Wenn du dieses Buch gelesen hast, wirst du
feststellen, dass Gigi der ultimative Bitcoin-Sherpa ist, und ich werde ihn für
immer zu schätzen wissen.

\begin{flushright}
  Hass McCook \\
  29. November 2019
\end{flushright}
