
\def\bitcoinB{\leavevmode
  {\setbox0=\hbox{\textsf{B}}%
    \dimen0\ht0 \advance\dimen0 0.2ex
    \ooalign{\hfil \box0\hfil\cr
      \hfil\vrule height \dimen0 depth.2ex\hfil\cr
    }%
  }%
}

\chapter*{About This Book (and About the Author)}
\pdfbookmark{About This Book (and About the Author)}{about}

This is a bit of an unusual book. But hey, Bitcoin is a bit of an unusual
technology, so an unusual book about Bitcoin might be fitting. I'm not sure if
I'm an unusual guy (I like to think of myself as a \textit{regular} guy) but the
story of how this book came to be, and how I came to be an author, is worth
telling.

First of all, I'm not an author. I'm an engineer. I didn't study writing. I
studied code and coding. Second of all, I never intended to write a book, let
alone a book about Bitcoin. Hell, I'm not even a native speaker.\footnote{The
reason why I'm writing these words in English is that my brain works in
mysterious ways. Whenever something technical comes up, it switches to English
mode.} I'm just a guy who caught the Bitcoin bug. Hard.

Who am \textit{I} to write a book about Bitcoin? That's a good question. The
short answer is easy: I'm Gigi, and I'm a bitcoiner. Nice to meet you.

The long answer is a bit more nuanced.

\paragraph{} My background is in computer science and software development. In a
previous life, I was part of a research group that tried to make computers think
and reason, among other things. In yet another previous life I wrote software
for automated passport processing and related stuff which is even scarier. I
know a thing or two about computers and our networked world, so I guess I have a
bit of a head-start to understand the technical side of Bitcoin. However, as I
try to outline in this book, the tech side of things is only a tiny sliver of
the beast which is Bitcoin. And every single one of these slivers is important.

This book came to be because of one simple question: \textit{\enquote{What have
you learned from Bitcoin?}} I tried to answer this question in a single tweet.
Then the tweet turned into a tweetstorm. The tweetstorm turned into an article.
The article turned into three articles. Three articles turned into 21 Lessons.
And 21 Lessons turned into a book. So I guess I'm just really bad at condensing
my thoughts into a single tweet.

\paragraph{} \textit{\enquote{Why write this book?}}, you might
ask. Again, there is a short and a long answer. The short answer is that I
simply had to. I was (and still am) \textit{possessed} by Bitcoin. I find it to
be endlessly fascinating. I can't seem to stop thinking about it and the
implications it will have on our global society. The long answer is that I
believe that Bitcoin is the single most important invention of our time, and
more people need to understand the nature of this particular beast. Bitcoin is
still one of the most misunderstood phenomena of our modern world, and it took
me years to fully realize the gravitas of this alien technology. Realizing what
Bitcoin is and how it will transform our society is a realization of utmost
profundity. I hope to plant the seeds which might lead to this realization in
your head, by writing these words.

While this section is titled \enquote{\textit{About This Book (and About the
Author)}}, in the grand scheme of things, this book, who I am, and what I did
doesn't really matter. I am just a node in the network, both literally
\textit{and} figuratively. Plus, you shouldn't trust what I'm saying anyway. As
we bitcoiners like to say: do your own research, and most importantly: don't
trust, verify.

I did my best to do my homework and provide plenty of sources for you, dear
reader, to dive into. In addition to the footnotes and citations in this book, I
try to keep an updated list of resources at
\href{https://21lessons.com/rabbithole}{21lessons.com/rabbithole} and on
\href{https://bitcoin-resources.com}{bitcoin-resources.com}, which also lists
plenty of other curated resources, books, and podcasts that will help you to
understand what Bitcoin is.

\paragraph{} This is simply a book about Bitcoin, written by a bitcoiner.
Bitcoin doesn't need this book, and you probably don't need this book to
understand Bitcoin. I believe that Bitcoin will be understood by you as soon as
\textit{you} are ready, and I also believe that the first fractions of a bitcoin
will find you as soon as you are ready to receive them. In essence, everyone
will get \bitcoinB{}itcoin at exactly the right time. In the meanwhile, Bitcoin
simply is, and that is enough.\footnote{Beautyon, \textit{Bitcoin Is. And That
Is Enough.} TODO CITE}
