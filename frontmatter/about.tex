
\def\bitcoinB{\leavevmode
  {\setbox0=\hbox{\textsf{B}}%
    \dimen0\ht0 \advance\dimen0 0.2ex
    \ooalign{\hfil \box0\hfil\cr
      \hfil\vrule height \dimen0 depth.2ex\hfil\cr
    }%
  }%
}

\chapter*{Über dieses Buch \\ (\ldots und über den Autor)}
\pdfbookmark{Über dieses Buch \\ (\ldots und über den Autor)}{about}

Dies ist ein etwas ungewöhnliches Buch. Aber hey, Bitcoin ist eine etwas
ungewöhnliche Technologie, daher ist ein ungewöhnliches Buch über Bitcoin wohl
passend. Ich bin mir nicht sicher, ob ich ein ungewöhnlicher Typ bin (ich
betrachte mich gerne als einen \textit{gewöhnlichen} Typen), aber die
Geschichte, wie dieses Buch entstand und wie es kam, dass ich es schrieb, ist es
wert, erzählt zu werden.

Erstens: ich bin kein Autor. Ich bin Ingenieur. Ich bin kein Experte was
Literatur oder die Schreiberei angeht. Ich habe mich meine ganze berufliche
Laufbahn lang nicht mit Wörtern, sondern mit Programmcode und Programmierung
auseinandergesetzt. Zweitens: ich wollte nie ein Buch schreiben, geschweige denn
ein Buch über Bitcoin, und das noch dazu auf Englisch! Englisch ist nichtmal
meine Muttersprache!\footnote{Warum habe ich dieses Buch dann ursprünglich auf
Englisch geschrieben? Ganz einfach: Aus irgendeinem mysteriösem Grund wechselt
mein Hirn auf Englisch sobald es um etwas technisches wie Bitcoin geht. Ich
hoffe allerdings dass die Deutsche Übersetzung lesbar und verständlich ist.} Ich
bin nur ein Typ, der vom Bitcoin-Fieber angesteckt wurde.

Warum sollte ausgerechnet \textit{ich} ein Buch über Bitcoin schreiben? Und wer
zum Teufel bin ich überhaupt? Das ist eine gute Frage. Die kurze Antwort ist
einfach: Ich bin der Gigi, und ich bin ein Bitcoiner. Die lange Antwort ist
etwas schwieriger.

\paragraph{}
Ich komme aus der Welt der Informatik und der Softwareentwicklung. In einem
früheren Leben war ich Teil einer Forschungsgruppe, die unter anderem versuchte,
Computern das \enquote{Denken} beizubringen (oder sie zumindest so weit zu
bringen, dass sie logische Schlüsse über unsere Welt ziehen können). Später habe
ich an verschiedenen beängstigenden Apps gearbeitet, wie z.B. automatisierte
Kontrolle und Verarbeitung von Passdaten. Ich weiß somit ein paar Dinge über
Computer und unsere vernetzte Welt, also habe ich wohl einen kleinen Vorsprung
wenn es darum geht die technische Seite von Bitcoin zu verstehen. Wie ich jedoch
in diesem Buch zu skizzieren versuche, ist die technische Komponente nur einer
von vielen  Teilen des Biests, welches sich Bitcoin nennt. Und jeder einzelne
dieser Teile ist wichtig.

Dieses Buch entstand aufgrund einer einfachen Frage: \textit{\enquote{Was hast
du von Bitcoin gelernt?}} Zuerst versuchte ich diese Frage in einem kurzen Tweet
zu beantworten. Dann verwandelte sich der Tweet in viele zusammenhängende
Tweets. Aus diesen wurde dann ein Artikel. Und aus dem Artikel wurden drei
Artikel, welche insgesamt 21 Lektionen beinhalteten. Aus diesen 21 Lektionen
wurde schlussendlich dieses Buch. Die Moral von der Geschichte? Ich bin, wie es
scheint,  unglaublich schlecht darin meine Gedanken in einem einzigen Tweet
zusammenzufassen.

\paragraph{}
\enquote{Warum dieses Buch schreiben?}, fragst du dich nun vielleicht. Wieder
gibt es eine kurze und eine lange Antwort. Die kurze Antwort ist, dass ich es
einfach schreiben musste. Ich war (und bin) von Bitcoin regelrecht besessen. Es
ist ein endlos faszinierendes Thema, und wie es scheint kann ich nicht aufhören
über die Technologie und die Auswirkungen welche Bitcoin auf unsere globale
Gesellschaft haben wird nachzudenken. Die lange Antwort ist, dass ich glaube,
dass Bitcoin die wichtigste Erfindung unserer Zeit ist und mehr Menschen die
Natur dieser Erfindung verstehen müssen. Bitcoin ist immer noch eines der am
missverstandensten Phänomene unserer modernen Welt, und es hat Jahre
gedauert, bis ich die Bedeutsamkeit dieser außerirdischen Technologie vollständig
erkannt habe. Zu erkennen, was Bitcoin ist und wie es unsere Gesellschaft
verändern wird, ist eine tiefgreifende Erfahrung. Ich hoffe, die Grundsteine, die zu
dieser Erkenntnis führen könnten, in diesem Buch zu umschreiben.

Auch wenn dieser Abschnitt \enquote{Über dieses Buch (\ldots und über den
Autor)} heißt, spielt es im Großen und Ganzen keine Rolle wer ich bin und was
ich mache. Ich bin nur ein Knoten im Netzwerk, sowohl wörtlich als auch im
übertragenen Sinne. Außerdem solltest du dem, was ich sage, sowieso nicht blind
vertrauen. Wie wir Bitcoiner gerne sagen: \textit{\enquote{don't trust, verify}}
-- recherchiere selbst und vor allem: nicht vertrauen, sondern überprüfen!

Ich habe mich bemüht meine Hausaufgaben zu machen und Euch, liebe Leser, viele
Quellenangaben zu bieten, in die man etwas tiefer eintauchen kann. Zusätzlich zu
den Fußnoten und Zitaten in diesem Buch versuche ich eine stets aktuelle Liste
dieser Quellen unter
\href{https://21lessons.com/rabbithole}{21lessons.com/rabbithole} und auf
\href{https://bitcoin-resources.com}{bitcoin-resources.com} zu führen, in der
viele ausgewählte Artikel, Bücher und Podcasts zu finden sind welche hilfreich
sind um Bitcoin zu verstehen.

\paragraph{}
Kurz gesagt, dies ist einfach ein Buch über Bitcoin, das von einem Bitcoiner
geschrieben wurde. Bitcoin benötigt dieses Buch nicht, und du benötigst dieses
Buch wahrscheinlich nicht, um Bitcoin zu verstehen. Ich glaube, dass Bitcoin
verstanden wird sobald man dazu bereit ist es zu verstehen. Und ich glaube auch,
dass die ersten Bruchteile eines Bitcoin zu dir finden werden sobald du dazu
bereit bist diese zu erhalten. Im Wesentlichen wird bei jedem, ganz individuell,
zum richtigen Zeitpunkt der orange Groschen fallen. In der Zwischenzeit
existiert Bitcoin einfach, und das ist genug.\footnote{Beautyon,
\textit{Bitcoin is. And that is enough.}~\cite{bitcoin-is}}
